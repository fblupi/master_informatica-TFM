\documentclass[a4paper,11pt]{book}
\usepackage{listings}
\usepackage{xspace}
\usepackage{url}
\usepackage[utf8]{inputenc}
\usepackage[spanish]{babel}
\usepackage{pdfpages}

\decimalpoint
\usepackage{dcolumn}
\newcolumntype{.}{D{.}{\esperiod}{-1}}
\makeatletter
\addto\shorthandsspanish{\let\esperiod\es@period@code}
\makeatother

\RequirePackage{verbatim}
\usepackage{fancyhdr}
\usepackage{graphics, graphicx, float}
\usepackage{afterpage}

\usepackage{longtable}

\usepackage[pdfborder={000}]{hyperref}

% ********************************************************************
% Re-usable information
% ********************************************************************
\newcommand{\myTitle}{3DCurator\xspace}
\newcommand{\mySubtitle}{Sistema gráfico de ayuda al diagnóstico e intervención de esculturas mediante datos médicos volumétricos\xspace}
\newcommand{\myEnglishSubtitle}{Graphic system to aid the diagnosis and intervention of sculptures using volumetric medical data\xspace}
\newcommand{\myDegree}{Máster en Ingeniería Informática\xspace}
\newcommand{\myName}{Francisco Javier Bolívar Lupiáñez\xspace}
\newcommand{\myProf}{Francisco Javier Melero Rus\xspace}
\newcommand{\myFaculty}{Escuela Técnica Superior de Ingenierías Informática y de Telecomunicación\xspace}
\newcommand{\myFacultyShort}{E.T.S. de Ingenierías Informática y de Telecomunicación\xspace}
\newcommand{\myDepartment}{Departamento de Lenguajes y Sistemas Informáticos\xspace}
\newcommand{\myUni}{\protect{Universidad de Granada}\xspace}
\newcommand{\myLocation}{Granada\xspace}
\newcommand{\myTime}{\today\xspace}


\hypersetup{
pdfauthor = {\myName (fblupi@correo.ugr.es)},
pdftitle = {\myTitle},
pdfsubject = {},
pdfkeywords = {3DCurator, informática gráfica, renderizado de volúmenes, tomografía computarizada, escultura policromada de madera, restauración, conservador de arte},
pdfproducer = {pdflatex}
}

\usepackage{url}
\usepackage{colortbl,longtable}
\usepackage[stable]{footmisc}

\pagestyle{fancy}
\fancyhf{}
\fancyhead[LO]{\leftmark}
\fancyhead[RE]{\rightmark}
\fancyhead[RO,LE]{\textbf{\thepage}}
\renewcommand{\chaptermark}[1]{\markboth{\textbf{#1}}{}}
\renewcommand{\sectionmark}[1]{\markright{\textbf{\thesection. #1}}}

\setlength{\headheight}{1.5\headheight}

\newcommand*\justify{
	\fontdimen2\font=0.4em
	\fontdimen3\font=0.2em
	\fontdimen4\font=0.1em
	\fontdimen7\font=0.1em
	\hyphenchar\font=`\-
}
\newcommand{\HRule}{\rule{\linewidth}{0.5mm}}

\definecolor{gray97}{gray}{.97}
\definecolor{gray75}{gray}{.75}
\definecolor{gray45}{gray}{.45}
\definecolor{gray30}{gray}{.94}

\lstset{ frame=Ltb,
     framerule=0.5pt,
     aboveskip=0.5cm,
     framextopmargin=3pt,
     framexbottommargin=3pt,
     framexleftmargin=0.1cm,
     framesep=0pt,
     rulesep=.4pt,
     backgroundcolor=\color{gray97},
     rulesepcolor=\color{black},
     stringstyle=\ttfamily,
     showstringspaces = false,
     basicstyle=\scriptsize\ttfamily,
     commentstyle=\color{gray45},
     keywordstyle=\bfseries,
     numbers=left,
     numbersep=6pt,
     numberstyle=\tiny,
     numberfirstline = false,
     breaklines=true,
}

\lstnewenvironment{listing}[1][]
   {\lstset{#1}\pagebreak[0]}{\pagebreak[0]}

\lstdefinestyle{C} {
	basicstyle=\scriptsize,
	frame=single,
	language=C,
	numbers=left
}

\lstdefinestyle{C++} {
	basicstyle=\small,
	frame=single,
	backgroundcolor=\color{gray30},
	language=C++,
	numbers=left
}

\lstdefinestyle{Consola} {
   	basicstyle=\scriptsize\bf\ttfamily,
    backgroundcolor=\color{gray30},
    frame=single,
    language=shell,
    numbers=none
}

\lstdefinestyle{XML} {
	basicstyle=\scriptsize,
	frame=single,
	language=XML,
	numbers=left
}

\newcommand{\bigrule}{\titlerule[0.5mm]}

\makeatletter
\def\clearpage{%
  \ifvmode
    \ifnum \@dbltopnum =\m@ne
      \ifdim \pagetotal <\topskip
        \hbox{}
      \fi
    \fi
  \fi
  \newpage
  \thispagestyle{empty}
  \write\m@ne{}
  \vbox{}
  \penalty -\@Mi
}
\makeatother

\usepackage{pdfpages}
\begin{document}
\input{portada/portada}
\chapter*{}

\input{portada/portada_2}

\cleardoublepage
\thispagestyle{empty}

\begin{center}
{\large\bfseries \myTitle: \mySubtitle}\\
\end{center}

\begin{center}
\myName \\
\end{center}

\vspace{0.7cm}
\noindent{\textbf{Palabras clave}: informática gráfica, volúmenes, filtros, segmentación, tomografía computarizada, escultura policromada de madera, conservación y restauración, restaurador de arte}\\

\vspace{0.7cm}
\noindent{\textbf{Resumen}}\\

Tras resolver el problema del renderizado de conjuntos de datos volumétricos y crear un software sencillo para estas tareas, se pretende continuar el desarrollo de éste enfocándose en otros tres objetivos claros. Dos de ellos son completar las otras dos fases de modelado de volúmenes previas al renderizado: el filtrado y la segmentación. Y el otro es proveer al software de herramientas para poder realizar tareas de documentación por parte de los usuarios finales de éste. En cuanto al filtrado, se van a proporcionar filtros básicos de reducción de ruido para no renderizas los datos en crudo que nos devuelve el escáner. El problema de segmentación que se tratará de resolver será el de dividir la figura en los distintos trozos de madera que la conforman. Por último se creará un entorno de documentación integrado en el software para poder realizar cualquier apunte dentro de la aplicación. Se integrarán widgets para realizar medidas de distancias y ángulos y para hacer anotaciones de texto en distintos cortes generados. Todas estas tareas se dividirán en distintas tareas y se seguirá un desarrollo ágil usando las herramientas que me proporciona GitHub para la gestión de un proyecto. Una vez terminado el desarrollo se realizarán una serie de pruebas y se documentarán los resultados obtenidos.

\thispagestyle{empty}

\cleardoublepage

\begin{center}
	{\large\bfseries \myTitle: \myEnglishSubtitle}\\
\end{center}

\begin{center}
	\myName \\
\end{center}

\vspace{0.7cm}
\noindent{\textbf{Keywords}: Computer Graphics, Volume, Filtering, Segmentation, Computed Tomography, Polychromed Wood Sculpture, Conservation-Restoration of Cultural Heritage, Art Curator}\\

\vspace{0.7cm}
\noindent{\textbf{Abstract}}\\

After solving the problem of volume rendering and build an application to do these tasks we will continue the development focusing on three well-differentiated objectives. Two of them are to complete the other two steps of volume modelling before the rendering: filtering and segmentation. And the other one is to provide the software of tools to make documentation tasks by the final user. To solve the filtering objective we will provide basic reduction noise filters to apply the volumetric dataset before rendering it. Regarding the segmentation, we will try to solve the problem of divide the original figure into the different pieces of wood it is composed by. Finally we will build an integrated documentation environment in the software. We will integrate widgets to make measurements of distances and angles and to make text annotations in different generated slices. All these tasks will be divided into sub-tasks and we will follow an agile development by using the tools provided by GitHub to manage a project. Once we finish the development, we will make some tests and we will documentate the obtained results.

\chapter*{}
\thispagestyle{empty}

\noindent\rule[-1ex]{\textwidth}{2pt}\\[4.5ex]

Yo, \textbf{\myName}, alumno de la titulación \myDegree de la \textbf{\myFaculty}, con DNI 75926571Y, autorizo la ubicación de la siguiente copia de mi Trabajo Fin de Máster en la biblioteca del centro para que pueda ser consultada por las personas que lo deseen.

\vspace{6cm}

\noindent Fdo: \myName

\vspace{2cm}

\begin{flushright}
\myLocation a \myTime.
\end{flushright}


\chapter*{}
\thispagestyle{empty}

\noindent\rule[-1ex]{\textwidth}{2pt}\\[4.5ex]

D. \textbf{\myProf}, Profesor del Área de Lenguajes y Sistemas Informáticos del \myDepartment de la \myUni.

\vspace{0.5cm}

\textbf{Informa:}

\vspace{0.5cm}

Que el presente trabajo, titulado \textit{\textbf{\myTitle, \mySubtitle}}, ha sido realizado bajo su supervisión por \textbf{\myName}, y autorizamos la defensa de dicho trabajo ante el tribunal que corresponda.

\vspace{0.5cm}

Y para que conste, expiden y firman el presente informe en \myLocation a \myTime.

\vspace{1cm}

\textbf{El director:}

\vspace{5cm}

\noindent \textbf{\myProf}

\chapter*{Agradecimientos}
\thispagestyle{empty}

\vspace{1cm}

En primer lugar, agradecer a todos aquellos que han hecho esto posible: 
\\

A mi tutor, Javier Melero, que confió en mi para este proyecto cuando hace ya casi tres años fui a su despacho en busca de un trabajo de fin de grado que ha podido ser continuado como trabajo fin de máster. 
\\

Al Portal Virtual de Patrimonio de las Universidades Andaluzas por la cesión de los modelos de las esculturas de Inmaculada Concepción y San Juan Evangelista que han sido utilizadas para probar el software. 
\\

Y a Concha y Amparo de Artemisia Gestión de Patrimonio por ofrecerme parte de su tiempo y proporcionar su conocimiento del tema.
\\

Por otra parte, también querría que tuviesen un apartado en esta sección ese grupo fantástico de compañeros que he tenido durante el máster. 
\\

Y cómo no, a todos los que me han animado a seguir cuando apenas quedaban fuerzas.
\frontmatter
\tableofcontents
\listoffigures
\listoftables

\mainmatter
\setlength{\parskip}{5pt}

\chapter{Introducción}

Este trabajo de fin de máster es una continuación del trabajo de fin de grado que realicé en el que se desarrolló un \textit{software} (3DCurator) para visualizar un conjunto de datos volumétricos, en formato DICOM, de esculturas de madera policromada.

Con este \textit{software}, los expertos en la materia como restauradores o historiadores del arte podían inspeccionar el interior de las esculturas sin dañarlas para un posterior proceso de estudio, restauración y/o conservación.

En este trabajo de fin de máster se llevarán a cabo distintas tareas de desarrollo que se integrarán a 3DCurator así como un estudio teórico más completo del proceso de obtención de datos volumétricos en los objetos que abarcan el campo de estudio en cuestión.

Las tareas de desarrollo que se integrarán con el \textit{software} desarrollado se dividen en tres bloques:

\begin{itemize}
	\item \textbf{Pre-procesamiento de datos}: Se estudiarán los distintos filtros disponibles para ver cuáles ofrecen mejores resultados en la tarea de reducción de ruido.
	\item \textbf{Subdivisión de piezas de madera}: Las esculturas suelen estar formadas por distintas piezas de madera. Se estudiará la forma de segmentarla probando, en primer lugar, los algoritmos ya existentes utilizados principalmente en medicina. Si ninguno ofrece los resultados que esperamos, se pasará a desarrollar uno propio.
	\item \textbf{Herramientas de documentación}: Se incluirán herramientas para ayudar a los usuarios en la tarea de documentación permitiendo, por ejemplo, incluir distintas anotaciones en puntos de interés.
\end{itemize}

Además de las librerías que ya se utilizaron: VTK \cite{vtk} (visualización), Qt \cite{qt} (GUI) y Boost \cite{boost} (XML); se utilizarán las librerías ITK \cite{itk} y OpenCV \cite{opencv} para el análisis de imágenes y la visión por computador respectivamente. Haciendo uso de CMake \cite{cmake} para pre-compilarlas todas juntas.

Antes de empezar a profundizar en aspectos técnicos se realizará una introducción al proceso de obtención de datos volumétricos usando una Tomografía Computarizada así como de los objetos que se quieren analizar con esta técnica de obtención de datos: las esculturas de madera policromada.

\section{Tomografía Computarizada}

\subsection{Historia}

La tomografía computarizada (TC) es una técnica de obtención de imágenes muy utilizado en el campo de la medicina para, por ejemplo, localizar y ver el tamaño de tumores.

Sus orígenes se remontan a los años 60 cuando en 1967 Goodfrey Newblod Hounsfield propuso la elaboración del que llamó escáner EMI, base para desarrollar el Tomógrafo Axial Computarizado (TAC). El objetivo era ``\textit{crear una imagen tridimensional de un objeto tomando múltiples mediciones del mismo con la misma fuente de rayos X desde diferentes ángulos y utilizar un ordenador que permita reconstruir a partir de cientos de `planos' superpuestos y entrecruzados}'' \cite{gonzales11}.

Cuatro años más tarde, en 1971, se realizó con éxito el primer escáner cerebral usando este tomógrafo. En 1972 se instaló permanentemente en el hospital donde realizaron las pruebas y al año siguiente ya era solicitado por hospitales alrededor de todo el mundo.

\subsection{Generaciones}

El sistema de tomografía computarizada ha pasado por cuatro generaciones \cite{sarrio16}:

\subsubsection{Primera generación}

La adquisición de datos en la primera generación se basaba en la geometría del haz de rayos X paralelo y traslación-rotación en un tubo de rayos X y un solo detector (Figura \ref{fig:introduccion/primera-generacion}). El haz de rayos X se colimaba en dimensiones de 2 x 13mm. Estos 13mm correspondían al grosor del corte. Se tomaba una medida por cada 160 rotaciones durante 180 traslaciones dando un total de 28.800 medidas. El proceso era lento, tardaba unos 5 minutos, y el movimiento del paciente afectaba muy negativamente a la calidad de la imagen, por lo que su uso se veía reducido al escaneo de zonas que podían mantenerse inmóviles como la cabeza.

\begin{figure}[H]
	\centering
	\includegraphics[width=10cm]{imagenes/introduccion/primera-generacion}
	\caption{Primera generación de aparatos de tomografía computarizada \cite{garcia14}}
	\label{fig:introduccion/primera-generacion}
\end{figure}

\subsubsection{Segunda generación}

En esta segunda generación se aumentó el número de detectores (de 5 a 30) por lo que se vio disminuido el tiempo de la exploración a unos 18 segundos (Figura \ref{fig:introduccion/segunda-generacion}).

\begin{figure}[H]
	\centering
	\includegraphics[width=10cm]{imagenes/introduccion/segunda-generacion}
	\caption{Segunda generación de aparatos de tomografía computarizada \cite{garcia14}}
	\label{fig:introduccion/segunda-generacion}
\end{figure}

\subsubsection{Tercera generación}

La tercera generación supuso un gran cambio y se ha convertido en la configuración estándar utilizada en la mayoría de los sistemas de escáner. Se utiliza una geometría de haz en abanico de gran angular (50º a 55º), un arco de detectores y un tubo de rayos X. Estos elementos giran 360º alrededor del paciente (Figura \ref{fig:introduccion/tercera-generacion}). El número de detectores se encuentra entre 600 y 900. Con este sistema el tiempo de barrido oscila entre 3 y 10 segundos.

\begin{figure}[H]
	\centering
	\includegraphics[width=6cm]{imagenes/introduccion/tercera-generacion}
	\caption{Tercera generación de aparatos de tomografía computarizada \cite{garcia14}}
	\label{fig:introduccion/tercera-generacion}
\end{figure}

\subsubsection{Cuarta generación}

La cuarta generación es muy parecida a la tercera solo que añade una configuración de giro estacionario (Figura \ref{fig:introduccion/cuarta-generacion}).

\begin{figure}[H]
	\centering
	\includegraphics[width=6cm]{imagenes/introduccion/cuarta-generacion}
	\caption{Cuarta generación de aparatos de tomografía computarizada \cite{garcia14}}
	\label{fig:introduccion/cuarta-generacion}
\end{figure}

\subsubsection{Nuevas tecnologías}

\begin{itemize}
	\item \textbf{TC helicoidal (TCH)}: Hasta finales de los años 80, los aparatos de TC adquirían los datos en cortes según un método conocido como exploración axial (de ahí el nombre de TAC). Con los sistemas de tipo helicoidal los datos se obtienen de forma continua mientras se avanza la mesa a través del \textit{gantry} haciendo que el tubo de rayos X describa una trayectoria helicoidal (Figura \ref{fig:introduccion/tch}).
	
	\begin{figure}[H]
		\centering
		\includegraphics[width=9cm]{imagenes/introduccion/tch}
		\caption{Cuarta generación de aparatos de tomografía computarizada \cite{garcia14}}
		\label{fig:introduccion/tch}
	\end{figure}
	
	\item \textbf{TC helicoidal multicorte (TCM)}: En el lugar donde había una fila de detectores, se colocan múltiples filas. Los primeros tenían 4 filas contiguas, pero posteriormente se ha pasado a alrededor de 16 y 64 filas (Figura \ref{fig:introduccion/tcm}). Por cada rotación se estudia un mayor volumen aumentando así la velocidad de rotación y por tanto los tiempos de exposición obteniendo imágenes de mayor calidad.
	
	\begin{figure}[H]
		\centering
		\includegraphics[width=10cm]{imagenes/introduccion/tcm}
		\caption{Diferencias entre TC helicoidal multicorte (B) y monocorte (A) \cite{sarrio16}}
		\label{fig:introduccion/tcm}
	\end{figure}

	\item \textbf{TC de doble fuente (TCED)}: Es uno de los equipos más novedosos pues permiten realizar estudios con diferentes espectros de rayos X. Utilizan dos tubos de rayos X colocados de forma perpendicular en el \textit{gantry} (Figura \ref{fig:introduccion/tced}). Se obtiene por tanto una resolución temporal equivalente a un cuarto del tiempo de rotación del \textit{gantry}.
	
	\begin{figure}[H]
		\centering
		\includegraphics[width=9cm]{imagenes/introduccion/tced}
		\caption{Equipo TC con doble fuente \cite{sarrio16}}
		\label{fig:introduccion/tced}
	\end{figure}

\end{itemize}



\section{Esculturas de madera policromadas}

\subsection{Historia}

El tallado es el método de elaboración de esculturas más antiguo conocido. Se ha tallado en distintos materiales (madera, piedra, marfil...). Pero la madera, por condiciones como su ligereza o la facilidad de ensamblado entre distintas piezas, ha sido uno de los materiales más utilizados.

Se conoce que desde el Antiguo Egipto ya se realizaban esculturas de madera pero es a partir del siglo XI cuando comienza su proliferación. Y desde este momento comienzan a producirse mejoras en las técnicas y herramientas utilizadas durante el proceso del tallado \cite{sarrio16}.

\subsection{Maderas más utilizadas}

Dependiendo del tipo de escultura se utilizan maderas blandas o duras. Si la escultura es más pequeña y contiene más detalles se utilizará un tipo de madera más dura.

No obstante, en la elección también tiene mucha influencia la situación geográfica al utilizarse maderas autóctonas \cite{sarrio16}:

\begin{itemize}
	\item \textbf{Italia}: Álamo y chopo.
	\item \textbf{Francia}: Nogal y castaño.
	\item \textbf{Países Bajos}: Roble y encina.
	\item \textbf{España}: Pino de Flandes, cedro de la Habana, castaño, tejo, álamo, nogal, ciprés, boj, pino silvestre y algunos frutales como el peral.
\end{itemize}

Además de la madera, una escultura de madera policromada, puede contener varios materiales como el estuco o el metal de los clavos utilizados.

\subsection{Defectos de la madera}

Entre los defectos de la madera se pueden encontrar \cite{sarrio16}:

\subsubsection{Grietas o fendas}
Según la UNE-EN 844-9 se denomina grieta o fenda a ``toda separación de las fibras (raja o hendidura) en dirección longitudinal". Según su origen, pueden ser de distintos tipos: 

\begin{itemize}
	\item \textbf{Acebolladuras o \textit{colainas}}: Hay una discontinuidad entre los anillos de crecimiento.
	\item \textbf{Superficiales o de desecación}: Producidas por el calor, provocan un deterioro en las zonas externas del tronco del árbol dejando la madera desprotegida. Provocan grietas en sentido longitudinal.
	\item \textbf{De heladura}: Producidas por una helada dañan la superficie e interior del tronco. Provocan grietas radiales.
	\item \textbf{De viento}: Originadas por la acción de un fuerte viento. Provocan grietas longitudinales y transversales.
\end{itemize}

Además de estos procesos naturales, se pueden producir grietas durante procesos como el secado que provocan una separación de las fibras.

\subsubsection{Fibras reviradas y entrelazadas}

Las fibras se encuentran normalmente orientadas en paralelo al eje principal del tronco, pero en ocasiones pueden presentar nudos que alteran la dirección de éstas.

\subsubsection{Nudos}

La UNE 56.521 define nudo como ``anomalía local de la estructura de la madera producida por la parte inferior de una rama que va quedando englobada en el tronco a medida que se producen los crecimientos de este". Existen distintos tipos:

\begin{itemize}
	\item \textbf{Adherente, vivo, fijo o sano}: Definido por la UNE 56.521 como ``aquel cuyos tejidos son solidarios con los de la madera que los rodea debido a ser formado por una rama viva".
	\item \textbf{Suelto, saltadizo, muerto o seco}: Definido por la UNE 56.521 como ``aquel en que los tejidos de la rama que lo producen no son solidarios con los de la madera que los rodea y suelen separarse".
\end{itemize}

\subsubsection{Núcleos de resina}

Son cavidades entre los anillos de crecimiento producidos frecuentemente por nudos.

\subsubsection{Factores de deterioro de tipo biótico}

Además de las alteraciones que ya presenta la madera, existen otros factores que también influyen como la humedad y la temperatura o el ataque de insectos xilófagos y hongos.

Los insectos xilófagos se nutren de madera seca y favorecen a su desarrollo una humedad relativa y temperaturas no muy bajas. Estos producen daños rompiendo las fibras de la madera.

Los hongos son microorganismos que pueden desarrollarse en la superficie o en el interior de la madera, haciendo que pierda humedad, reduciendo su tamaño y deformándose. Existen tres tipos distintos de degradación tras un ataque de hongos: pudrición blanca, parda o seca y blanda.

\subsection{Proceso de tallado}

El proceso de tallado se podría definir como una técnica sustractiva en la que a partir de una pieza se obtiene una forma concreta.

Un método puede ser utilizar un único bloque de madera. En ocasiones ahuecado por el reverso para contrarrestar fuerza y movimiento de la madera, y, en cierto modo, aligerando el peso.

A partir del siglo XVI empieza a utilizarse con más frecuencia otro método en el que a partir de un bloque principal se ensamblan diferentes piezas generando un bloque más grande, denominado embón, con la forma y el tamaño de la imagen a esculpir. A partir de éste se comienza el tallado.

A partir del Barroco se mejora esta técnica realizando ensamblados en hueco para evitar el posterior ahuecado \cite{sarrio16}.

\section{Estado del arte}

Como ya se ha comentado anteriormente, este trabajo es la continuación del trabajo fin de grado que realicé. Durante este trabajo fin de grado resolví el problema de la visualización de datos volumétricos de esculturas de madera policromada \cite{bolivar16}. Pero no realizaba ningún pre-procesamiento, tan solo se visualizaban permitiendo explorar la escultura en su interior. Toda la información que obtenía el experto que usase el programa debía ser anotada a mano a parte pues tampoco se proporcionaba funcionalidad para almacenar esta información para poder ser rescatada posteriormente.

Este trabajo se divide en tres bloques: pre-procesamiento, documentación y segmentación. 

Los dos primeros están incluidos en la mayoría de los programas disponibles en la web de visualización de datos volumétricos. Tales como AMILab \cite{krissian12}, RadiAnt \cite{radiant}, Slicer \cite{fedorov12} o el usado por María Francisca en su estudio \cite{sarrio16} OsiriX \cite{rosset04} en el que me he inspirado para realizar mi implementación. Todas estas herramientas son muy completas, pero los resultados obtenidos al visualizar los datos volumétricos de esculturas de madera policromadas dejaban mucho que desear y el gran número de funciones específicas para la medicina los hacen complejos de más, y es que por esta razón surgió la idea de crear un software específico de visualización de datos volumétricos de esculturas.

El tercer bloque, el de la segmentación, es totalmente novedoso pues no hay ningún tipo de segmentación específica para separar distintos trozos de madera. Sin embargo, antes de programar un algoritmo específico habría que probar con los que ya hay disponibles que además son muy usados en el campo de la medicina.

Existen varias aproximaciones para llevar a cabo la segmentación en datos volumétricos. Pueden clasificarse en:

\begin{itemize}
	\item Segmentación manual
	\item Segmentación semi-automática
	\item Segmentación basada en umbrales (\textit{thresholding})
	\item Segmentación basada en crecimiento de regiones (\textit{region-growing})
	\item Segmentación basada en bordes (\textit{edge})
\end{itemize}

La segmentación manual la podríamos descartar automáticamente. Pese a que es un método que siempre se puede aplicar, tener a una persona recortando manualmente las distintas piezas corte por corte puede ser muy costoso en cuanto a tiempo.

Los métodos de segmentación basada en umbrales \cite{otsu79} se basan en definir un umbral en los valores escalares de cada vóxel para diferenciar las distintas partes del volumen. Estos no nos son útiles pues las distintas piezas de madera suelen ser del mismo material y coincidir en estos valores de densidad.

Los métodos de segmentación basada en crecimiento de regiones que utilizan umbrales \cite{haralick85} tampoco nos van a servir pues las distintas piezas de madera se encuentran juntas. Otros métodos de segmentación basada en regiones como la transformación divisoria (\textit{watershed}) \cite{beucher79} tampoco nos van a servir pues se definirían muchas cuencas debido a los anillos que presentan los cortes de la madera y si se empiezan a inundar para obtener regiones más grandes obtendremos un resultado similar al que obtenemos con el de crecimiento de regiones con umbrales.

Los métodos de segmentación basados en bordes como el \textit{livewire} \cite{mortensen95} podrían resultar efectivos pues el borde que separa dos piezas de madera es visualmente diferenciable. No obstante, este método necesita de una supervisión humana y, aunque sea más preciso y rápido que una segmentación manual, se debería seguir realizando corte a corte.

Como vemos, ninguno de los métodos que no requieren de supervisión humana son efectivos a la hora de resolver nuestro problema. Es por ello que, definitivamente, hay que desarrollar uno propio.

\section{Motivación}

La exploración interna de una escultura mediante los datos obtenidos con una TC \ref{fig:introduccion/ct-scan} son mucho más completos que los obtenidos con otras técnicas como las radiografías tradicionales \cite{sarrio16}. La necesidad de una herramienta con la que poder visualizar estos datos se vio satisfecha con 3DCurator \cite{bolivar16}, sin embargo, se trataba de un software todavía incompleto pues no contaba con técnicas de pre-procesamiento ni documentación como otros software del mercado especializados en visualizar datos médicos.

La extracción de los distintos trozos de madera que forman parte de la escultura podrán ayudar enormemente a su estudio por separado además de poder ser extraídos en un formato como el STL para ser imprimidos en 3D.

\begin{figure}[H]
	\centering
	\includegraphics[width=10cm]{imagenes/introduccion/ct-scan}
	\caption{Obtención de datos volumétricos de una escultura posteriormente examinada}
	\label{fig:introduccion/ct-scan}
\end{figure}
\chapter{Especificación de requisitos}

Este capítulo es una Especificación de Requisitos Software para el software que se va a realizar siguiendo las directrices dadas por el estándar IEEE830 \cite{iee830}.

\section{Introducción}

\subsection{Propósito}

Este capítulo de especificación de requisitos tiene como objetivo definir las especificaciones funcionales y no funcionales para el desarrollo de un software que permitirá pre-procesar datos DICOM obtenidos al someter a una escultura de madera policromada a una TC para que posteriormente los usuarios puedan realizar labores de documentación al explorar la figura así como realizar una segmentación de los distintos trozos de madera que la componen. 
\chapter{Planificación}

En este capítulo se comentará la planificación inicial de tiempo en la que se llevará a cabo este TFM, la estimación en horas de cada módulo de tareas, la metodología de trabajo y los resultados.

\section{Planificación inicial}

La fecha de inicio trabajando en el proyecto que me fijé es el 1 de diciembre de 2016 y se espera llegar a la convocatoria de julio de 2017 por lo que se cuentan con siete meses. No obstante, y como es lógico, no voy a trabajar exclusivamente en este proyecto durante estos meses pues tengo que cursar el resto de asignaturas del máster, además debería tener un mes de margen para pulir la memoria desarrollando algunas secciones y organizando lo que haya ido apuntando durante el resto de meses. Por lo que serían 6 meses de desarrollo.

Para organizarme he decidido dedicar un número de horas semanales al desarrollo de este proyecto. Inicialmente me he fijado 15 horas a la semana dando un total de 360 horas de desarrollo puro excluyendo la redacción de la memoria.

Ordenando los módulos por prioridades y dependencias se tendría la siguiente planificación:

\begin{longtable} {l c c c}
	\hline
	\textbf{Tarea}			&	\textbf{Duración}	&	\textbf{Comienzo}	&	\textbf{Fin}	\\
	\hline \hline
	\endhead
	\hline 
	Exportación/importación	&	2 semanas			&	01/12/2016			&	14/12/2016		\\
	\hline
	Filtrado				&	6 semanas			&	15/12/2016			&	25/01/2017		\\
	\hline
	Segmentación			&	10 semanas			&	26/01/2017			&	05/04/2017		\\
	\hline
	Documentación			&	8 semanas			&	06/04/2017			&	31/05/2017		\\
	\hline
	Mejoras					&	1 semana			&	01/06/2017			&	07/06/2017		\\
	\hline
	Memoria					&	4 semanas			&	08/06/2017			&	05/07/2017		\\
	\hline
	\\
	\caption{Planificación inicial}
	\label{tab:planificacion/planificacion-inicial}
\end{longtable}

\subsection{Diagrama de Gantt}

\begin{figure}[H]
	\centering
	\includegraphics[width=12cm]{imagenes/planificacion/gantt}
	\caption{Diagrama de Gantt con la planficación inicial del proyecto generado con \textit{Microsoft Office Project 2016}}
	\label{fig:planificacion/gantt}
\end{figure}


\section{Metodología de trabajo}

\subsection{\textit{GitHub}}

Aprovechando que se está usando \textit{git} como sistema de control de versiones en un repositorio de \textit{GitHub}, voy a aprovechar todos los recursos que nos ofrece esta plataforma para organizarme. 

\subsection{\textit{Issues}}

En lugar de tener por un lado un tablero \textit{kanban} (por ejemplo \textit{Trello}), yo voy a utilizar los \textit{issues} (Figura \ref{fig:planificacion/issues}) de mi repositorio en \textit{GitHub} para organizar las tareas, así como ir añadiendo tareas nuevas que vaya necesitando y tener un registro de qué he hecho y cuándo, que problemas han surgido, etc.

\begin{figure}[H]
	\centering
	\includegraphics[width=12cm]{imagenes/planificacion/issues}
	\caption{Vista con la lista de \textit{issues} abiertos con sus correspondientes etiquetas y el \textit{milestone} al que pertenecen}
	\label{fig:planificacion/issues}
\end{figure}

Una de las ventajas de utilizar los \textit{issues} de \textit{GitHub} antes que otra plataforma como \textit{Trello} es la posibilidad de referenciarlos en los \textit{commit} agregando \texttt{\#ref} en el mensaje. Y dentro del propio \textit{issue} se vería una lista de todos los \textit{commit} donde se ha trabajado en éste. Además, si antes de escribir la referencia del \textit{issue} se añade \texttt{closes}, el \textit{issue} se cerrará automáticamente sin necesidad de hacerlo de forma manual.

\subsection{\textit{Milestones}}

Voy a clasificar los \textit{issues} por \textit{milestones} (Figura \ref{fig:planificacion/milestones}), uno por cada módulo y al mismo tiempo voy a agregarles \textit{tags} para etiquetarlos según sean \textit{bugs}, mejoras, bloqueantes por necesidad de recursos o por ayuda necesitada, etc.

\begin{figure}[H]
	\centering
	\includegraphics[width=12cm]{imagenes/planificacion/milestones}
	\caption{Resumen de los \textit{milestones} con su porcentaje de elaboración}
	\label{fig:planificacion/milestones}
\end{figure}

\subsection{\textit{Branches}}

La organización que voy a llevar en cuanto a las ramas (\textit{branches}), es la siguiente, basada en \textit{Gitflow} \cite{gitflow}:

\begin{itemize}
	\item \textbf{\textit{master}}: Esta rama tendrá la última \textit{release} del programa. Debe ser completamente estable, pues es de donde se generan los instaladores finales de la aplicación.
	\item \textbf{\textit{develop}}: Esta es la rama principal con los últimos cambios desarrollados (Figura \ref{fig:planificacion/commits}). Cuando se tiene una batería de cambios suficiente como para generar una nueva versión del \textit{software} se realizará un \textit{pull request} a \textit{master} por lo que esta rama debe ser también estable.
	\item \textbf{\textit{features}}: Estas son ramas de trabajo que no tienen por qué ser estables. Por ejemplo, si se va a agregar un filtro \textit{gaussiano} se podría crear una rama con el nombre \textit{feature/gaussian-filter} y trabajar en esta rama hasta que se validase esta nueva funcionalidad. Entonces se haría un \textit{rebase} con \textit{develop} por si ha habido algún cambio durante el transcurso de este desarrollo para resolver conflictos y finalmente un \textit{pull request} a \textit{develop} para mezclar los cambios.
\end{itemize}

Además cuento con otras ramas auxiliares que no intervienen en el flujo de desarrollo pero donde almaceno información útil.

\begin{itemize}
	\item \textbf{\textit{design}}: En esta rama se encuentran los diagramas de paquetes y clases actualizados con la versión del software en la rama \textit{develop}.
	\item \textbf{\textit{assets}}: En esta rama se almacenan recursos como logos e imágenes utilizados en la aplicación.
	\item \textbf{\textit{user-manual}}: En esta rama se encuentra el manual de usuario de la aplicación.
	\item \textbf{\textit{gh-pages}}: En esta rama se encuentra la \textit{landing page} de la aplicación. Pues \textit{GitHub} ofrece gratuitamente alojamiento para una página web estática por repositorio.
\end{itemize}

\begin{figure}[H]
	\centering
	\includegraphics[width=12cm]{imagenes/planificacion/commits}
	\caption{Ejemplo de \textit{commits} en la rama \textit{develop}}
	\label{fig:planificacion/commits}
\end{figure}

\section{Resultados}

Los resultados no tienen nada que ver con la planificación inicial que se hizo. La carga del trabajo del resto de asignaturas del máster han hecho que le dedicase muchas menos horas a la semana que las que tenía pensado y durante algunos periodos lo he tenido que tener apartado.

Cuando llegó junio y vi que no llegaba a la convocatoria, recalculé tiempos para llegar a septiembre. Pero otra vez se vieron trastocados pues en agosto empecé a trabajar a jornada completa y las horas que le pude echar se volvieron a ver reducidas. No obstante llegué a tenerlo todo a excepción de la documentación. Que ya, una vez matriculado de nuevo en la asignatura la tomé con calma.

Además, me permití el lujo de añadir a última hora internacionalización al proyecto para poder tener la aplicación también en inglés además de la versión ya existente en español.

\begin{figure}[H]
	\centering
	\includegraphics[width=12cm]{imagenes/planificacion/resultados}
	\caption{Gráfico de los \textit{commits} realizados en la rama \textit{develop}. El gráfico es orientativo porque hay \textit{commits} con poca funcionalidad y otros con mucha, pero ayuda a ver las etapas donde más se trabajó}
	\label{fig:planificacion/resultados}
\end{figure}

En total he registrado 312 horas de desarrollo frente a las 360 planificadas. Esto quiere decir que la planificación de tareas no ha fallado como tal. La que ha fallado ha sido la planificación de disponibilidad de recursos humanos.
\chapter{Análisis}

En este capítulo se describirán las distintas historias de usuario que han sido implementadas en el software.

\section{Historias de usuario}

\subsection{\textit{Product backlog}}

A continuación se muestra el listado de historias de usuario (\textit{Product Backlog}) completo separadas por módulo, y para cada historia de usuario sus dependencias, estimación (en puntos de historia) y prioridad.

\begin{longtable} {r l c c c}
	\hline
	\#	&	\textbf{Descripción}					&	\textbf{Dep.}	&	\textbf{Est.}	&	\textbf{Prio.}	\\
	\hline \hline
	\endhead
	\multicolumn{5}{l}{\textbf{Auxiliar}} \\
	\hline 
	1.1.	&	Exportar volumen					&	-				&	12				&	1	\\
	\hline
	1.2.	&	Importar volumen					&	1.1.			&	8				&	5	\\
	\hline
	\multicolumn{5}{l}{\textbf{Pre-procesamiento}} \\
	\hline 
	2.1.	&	Filtro \textit{gaussiano}			&	-				&	24				&	1	\\
	\hline
	2.2.	&	Filtro media						&	-				&	10				&	3	\\
	\hline
	2.3.	&	Filtro mediana						&	-				&	8				&	3	\\
	\hline
	\multicolumn{5}{l}{\textbf{Segmentación}} \\
	\hline 
	3.1.	&	Segmentar pieza de madera			&	1.1.			&	64				&	1	\\
	\hline
	\multicolumn{5}{l}{\textbf{Documentación}} \\
	\hline 
	4.1.	&	Crear ROD							&	-				&	4				&	1	\\
	\hline
	4.2.	&	Eliminar ROD						&	4.1.			&	4				&	3	\\
	\hline
	4.3.	&	Exportar ROD						&	4.1.			&	2				&	2	\\
	\hline
	4.4.	&	Importar ROD						&	4.3.			&	3				&	2	\\
	\hline
	4.5.	&	Cambiar ROD							&	4.1.			&	3				&	1	\\
	\hline
	4.6.	&	Crear regla							&	-				&	3				&	2	\\
	\hline
	4.7.	&	Eliminar regla						&	4.6				&	3				&	4	\\
	\hline
	4.8.	&	Editar regla						&	4.6				&	1				&	5	\\
	\hline
	4.9.	&	Ocultar regla						&	4.6				&	2				&	6	\\
	\hline
	4.10.	&	Mostrar regla						&	4.9				&	1				&	6	\\
	\hline
	4.11.	&	Crear transportador de ángulos		&	-				&	3				&	3	\\
	\hline
	4.12.	&	Eliminar transportador de ángulos	&	4.11.			&	3				&	5	\\
	\hline
	4.13.	&	Editar transportador de ángulos		&	4.11.			&	1				&	6	\\
	\hline
	4.14.	&	Ocultar transportador de ángulos	&	4.11.			&	2				&	7	\\
	\hline
	4.15.	&	Mostrar transportador de ángulos	&	4.14.			&	1				&	7	\\
	\hline
	4.16.	&	Crear nota							&	-				&	4				&	2	\\
	\hline
	4.17.	&	Eliminar nota						&	4.16.			&	2				&	4	\\
	\hline
	4.18.	&	Editar nota							&	4.16.			&	3				&	5	\\
	\hline
	4.19.	&	Ocultar nota						&	4.16.			&	2				&	6	\\
	\hline
	4.20.	&	Mostrar nota						&	4.19.			&	1				&	6	\\
	\hline
	\multicolumn{5}{l}{\textbf{Mejoras en código}} \\
	\hline
	5.1.	&	Internacionalización				&	-				&	6				&	8	\\
	\hline
	\\
	\caption{Historias de usuario}
	\label{tab:analisis/hus}
\end{longtable}

Se han estimado un total de 180 PH (Puntos de historia) y se habían planificado unas 360 horas de trabajo. Por lo que cada PH corresponde aproximadamente a 2 horas.

Al haber registrado 312 horas finalmente, he determinado que mi velocidad ha sido aproximadamente 1,73 horas/PH, un poco más rápido que lo planificado.

\subsection{Tarjetas de las historias de usuario}

A continuación se incluye una descripción completa de las historias de usuario incluyendo una descripción de ésta y sus correspondientes criterios de aceptación.

\subsubsection{Auxiliar}

\begin{table}[H]
	\begin{center}
		\begin{tabular} {l|c|l}
			\hline
			1.1. & \multicolumn{2}{c}{Exportar volumen} \\ \noalign{\hrule height 1pt}
			\multicolumn{3}{l}{Descripción} \\ \hline
			\multicolumn{3}{p{12cm}}{Se debe proveer a la aplicación de la funcionalidad necesaria para exportar el volumen que se ha cargado y editado para poder utilizarlo de nuevo posteriormente sin tener que volver a editarlo. Para ello se tratará de utilizar el formato propio de VTK para almacenar volúmenes, VTI, que hace uso de XML. Se le mostrará al usuario un diálogo donde elegirá la carpeta y el nombre del archivo.} \\ \noalign{\hrule height 1pt}
			\multicolumn{2}{l|}{Estimación} & 12 \\ \hline
			\multicolumn{2}{l|}{Prioridad} & 1 \\ \hline
			\multicolumn{2}{l|}{Dependencias} & - \\ \noalign{\hrule height 1pt}
			\multicolumn{3}{l}{Pruebas de aceptación} \\ \hline
			\multicolumn{3}{p{12cm}}{ - El usuario realiza la acción de exportar sin ningún volumen cargado y le aparece un mensaje para que lo cargue antes.} \\
			\multicolumn{3}{p{12cm}}{ - El usuario no elige nombre y se guardará el archivo con un nombre por defecto ne la carpeta elegida.} \\ 
			\multicolumn{3}{p{12cm}}{ - El usuario elige el nombre y se guardará el archivo con el nombre elegido en la carpeta elegida.} \\ 
			\hline
		\end{tabular}
	\end{center}
	\caption{Historia de usuario - Exportar volumen}
	\label{tab:analisis/hu-exportar-volumen}
\end{table}

\begin{table}[H]
	\begin{center}
		\begin{tabular} {l|c|l}
			\hline
			1.2. & \multicolumn{2}{c}{Importar volumen} \\ \noalign{\hrule height 1pt}
			\multicolumn{3}{l}{Descripción} \\ \hline
			\multicolumn{3}{p{12cm}}{El \textit{software} debe poder leer ficheros en formato VTI como los previamente exportados. Para ello el usuario deberá poder elegir un archivo en un diálogo donde se filtrarán los ficheros para que se muestren solo los que tienen la extensión VTI.} \\ \noalign{\hrule height 1pt}
			\multicolumn{2}{l|}{Estimación} & 8 \\ \hline
			\multicolumn{2}{l|}{Prioridad} & 5 \\ \hline
			\multicolumn{2}{l|}{Dependencias} & 1.1. \\ \noalign{\hrule height 1pt}
			\multicolumn{3}{l}{Pruebas de aceptación} \\ \hline
			\multicolumn{3}{p{12cm}}{ - Si el usuario lee un archivo que no es VTI no lo podrá visualizar.} \\
			\multicolumn{3}{p{12cm}}{ - Si el usuario elige un archivo VTI correcto se importará y podrá empezar a utilizarlo.} \\ \hline
		\end{tabular}
	\end{center}
	\caption{Historia de usuario - Importar volumen}
	\label{tab:analisis/hu-importar-volumen}
\end{table}

\subsubsection{Pre-procesamiento}

\begin{table}[H]
	\begin{center}
		\begin{tabular} {l|c|l}
			\hline
			2.1. & \multicolumn{2}{c}{Filtro \textit{gaussiano}} \\ \noalign{\hrule height 1pt}
			\multicolumn{3}{l}{Descripción} \\ \hline
			\multicolumn{3}{p{12cm}}{Se debe poder aplicar un filtro \textit{gaussiano} al volumen para reducir el ruido. Para ello el usuario elegirá el número de repeticiones que se realizarán (de 1 a 5). Se utilizará la librería ITK para aplicar este filtro.} \\ \noalign{\hrule height 1pt}
			\multicolumn{2}{l|}{Estimación} & 24 \\ \hline
			\multicolumn{2}{l|}{Prioridad} & 1 \\ \hline
			\multicolumn{2}{l|}{Dependencias} & - \\ \noalign{\hrule height 1pt}
			\multicolumn{3}{l}{Pruebas de aceptación} \\ \hline
			\multicolumn{3}{p{12cm}}{ - El usuario filtra sin haber ningún volumen cargado y le aparece un mensaje de que cargue antes un modelo.} \\
			\multicolumn{3}{p{12cm}}{ - El usuario selecciona el filtrado y los parámetros deseados y se realiza el filtrado en el volumen, el resultado será un volumen más suavizado.} \\ \hline
		\end{tabular}
	\end{center}
	\caption{Historia de usuario - Filtro \textit{gaussiano}}
	\label{tab:analisis/hu-filtro-gaussiano}
\end{table}

\begin{table}[H]
	\begin{center}
		\begin{tabular} {l|c|l}
			\hline
			2.2. & \multicolumn{2}{c}{Filtro media} \\ \noalign{\hrule height 1pt}
			\multicolumn{3}{l}{Descripción} \\ \hline
			\multicolumn{3}{p{12cm}}{Se debe poder aplicar un filtro media al volumen para reducir ruido. Para ello el usuario elegirá el tamaño del vecindario (3x3, 5x5 o 7x7). Se utilizará la librería ITK para aplicar este filtro.} \\ \noalign{\hrule height 1pt}
			\multicolumn{2}{l|}{Estimación} & 10 \\ \hline
			\multicolumn{2}{l|}{Prioridad} & 3 \\ \hline
			\multicolumn{2}{l|}{Dependencias} & - \\ \noalign{\hrule height 1pt}
			\multicolumn{3}{l}{Pruebas de aceptación} \\ \hline
			\multicolumn{3}{p{12cm}}{ - El usuario filtra sin haber ningún volumen cargado y le aparece un mensaje de que cargue antes un modelo.} \\
			\multicolumn{3}{p{12cm}}{ - El usuario selecciona el filtrado y los parámetros deseados y se realiza el filtrado en el volumen, el resultado será un volumen más suavizado.} \\ \hline
		\end{tabular}
	\end{center}
	\caption{Historia de usuario - Filtro media}
	\label{tab:analisis/hu-filtro-media}
\end{table}

\begin{table}[H]
	\begin{center}
		\begin{tabular} {l|c|l}
			\hline
			2.3. & \multicolumn{2}{c}{Filtro mediana} \\ \noalign{\hrule height 1pt}
			\multicolumn{3}{l}{Descripción} \\ \hline
			\multicolumn{3}{p{12cm}}{Se debe poder aplicar un filtro mediana al volumen para reducir ruido. Para ello el usuario elegirá el tamaño del vecindario (3x3, 5x5 o 7x7). Se utilizará la librería ITK para aplicar este filtro.} \\ \noalign{\hrule height 1pt}
			\multicolumn{2}{l|}{Estimación} & 8 \\ \hline
			\multicolumn{2}{l|}{Prioridad} & 3 \\ \hline
			\multicolumn{2}{l|}{Dependencias} & - \\ \noalign{\hrule height 1pt}
			\multicolumn{3}{l}{Pruebas de aceptación} \\ \hline
			\multicolumn{3}{p{12cm}}{ - El usuario filtra sin haber ningún volumen cargado y le aparece un mensaje de que cargue antes un modelo.} \\
			\multicolumn{3}{p{12cm}}{ - El usuario selecciona el filtrado y los parámetros deseados y se realiza el filtrado en el volumen. Esto reducirá el ruido de tipo \textit{salt-and-pepper}.} \\ \hline
		\end{tabular}
	\end{center}
	\caption{Historia de usuario - Filtro mediana}
	\label{tab:analisis/hu-filtro-mediana}
\end{table}

\subsubsection{Segmentación}

\begin{table}[H]
	\begin{center}
		\begin{tabular} {l|c|l}
			\hline
			3.1. & \multicolumn{2}{c}{Segmentar pieza de madera} \\ \noalign{\hrule height 1pt}
			\multicolumn{3}{l}{Descripción} \\ \hline
			\multicolumn{3}{p{12cm}}{El usuario debe poder separar las distintas piezas de madera de las que está compuesta la figura. Para ello se desarrollará un método semi-automático guiado por el usuario para partir en dos un volumen, de forma iterativa se podría realizar la descomposición total de la figura. El usuario elegirá en el visor de corte el trozo deseado a segmentar y el sistema le preguntará por cuál de las posibles líneas detectadas divide las piezas de madera, el usuario le responderá y el sistema terminará el proceso. Para realizar esto se deberán combinar filtros, técnicas de visión por computador y de segmentación de volúmenes.} \\ \noalign{\hrule height 1pt}
			\multicolumn{2}{l|}{Estimación} & 64 \\ \hline
			\multicolumn{2}{l|}{Prioridad} & 1 \\ \hline
			\multicolumn{2}{l|}{Dependencias} & 1.1. \\ \noalign{\hrule height 1pt}
			\multicolumn{3}{l}{Pruebas de aceptación} \\ \hline
			\multicolumn{3}{p{12cm}}{ - El usuario elige una pieza de madera y la línea que la separa y se devolverá un volumen con esta pieza.} \\ \hline
		\end{tabular}
	\end{center}
	\caption{Historia de usuario - Segmentar pieza de madera}
	\label{tab:analisis/hu-segmentar-pieza-de-madera}
\end{table}

\subsubsection{Documentación}

\begin{table}[H]
	\begin{center}
		\begin{tabular} {l|c|l}
			\hline
			4.1. & \multicolumn{2}{c}{Crear ROD} \\ \noalign{\hrule height 1pt}
			\multicolumn{3}{l}{Descripción} \\ \hline
			\multicolumn{3}{p{12cm}}{Se debe poder crear un área de trabajo de documentación, es decir, imágenes donde incluir anotaciones. Para ello se ha creado el concepto ROD (Region of Documentation), que no es más que una posición del plano de corte donde se podrán incluir medidas y anotaciones. El usuario colocará el plano en la posición deseada y creará una ROD a la que podrá dar nombre, si no, cogerá un nombre por defecto.} \\ \noalign{\hrule height 1pt}
			\multicolumn{2}{l|}{Estimación} & 4 \\ \hline
			\multicolumn{2}{l|}{Prioridad} & 1 \\ \hline
			\multicolumn{2}{l|}{Dependencias} & - \\ \noalign{\hrule height 1pt}
			\multicolumn{3}{l}{Pruebas de aceptación} \\ \hline
			\multicolumn{3}{p{12cm}}{ - El usuario crea una ROD sin nombre y se guarda la posición del plano con un nombre por defecto.} \\
			\multicolumn{3}{p{12cm}}{ - El usuario crea una ROD con nombre y se guarda la posición del plano con el nombre indicado.} \\ \hline
		\end{tabular}
	\end{center}
	\caption{Historia de usuario - Crear ROD}
	\label{tab:analisis/hu-crear-rod}
\end{table}

\begin{table}[H]
	\begin{center}
		\begin{tabular} {l|c|l}
			\hline
			4.2. & \multicolumn{2}{c}{Eliminar ROD} \\ \noalign{\hrule height 1pt}
			\multicolumn{3}{l}{Descripción} \\ \hline
			\multicolumn{3}{p{12cm}}{El usuario debe poder eliminar una ROD creada con anterioridad. Para ello la selecciona y pulsa en la opción de eliminar.} \\ \noalign{\hrule height 1pt}
			\multicolumn{2}{l|}{Estimación} & 4 \\ \hline
			\multicolumn{2}{l|}{Prioridad} & 3 \\ \hline
			\multicolumn{2}{l|}{Dependencias} & 4.1. \\ \noalign{\hrule height 1pt}
			\multicolumn{3}{l}{Pruebas de aceptación} \\ \hline
			\multicolumn{3}{p{12cm}}{ - No hay ninguna ROD seleccionada, el usuario pulsa en eliminar y le aparece un mensaje de que debe seleccionar antes una.} \\
			\multicolumn{3}{p{12cm}}{ - El usuario selecciona una ROD, la elimina y desaparece del listado de ROD.} \\ \hline
		\end{tabular}
	\end{center}
	\caption{Historia de usuario - Eliminar ROD}
	\label{tab:analisis/hu-eliminar-rod}
\end{table}

\begin{table}[H]
	\begin{center}
		\begin{tabular} {l|c|l}
			\hline
			4.3. & \multicolumn{2}{c}{Exportar ROD} \\ \noalign{\hrule height 1pt}
			\multicolumn{3}{l}{Descripción} \\ \hline
			\multicolumn{3}{p{12cm}}{Se tiene que proveer al sistema de la funcionalidad necesaria para poder exportar una ROD con todos sus componentes de forma que pueda volver a importarlo. Para ello se exportará en formato XML.} \\ \noalign{\hrule height 1pt}
			\multicolumn{2}{l|}{Estimación} & 2 \\ \hline
			\multicolumn{2}{l|}{Prioridad} & 2 \\ \hline
			\multicolumn{2}{l|}{Dependencias} & 4.1. \\ \noalign{\hrule height 1pt}
			\multicolumn{3}{l}{Pruebas de aceptación} \\ \hline
			\multicolumn{3}{p{12cm}}{ - No hay ninguna ROD seleccionada, el usuario pulsa en exportar y le aparece un mensaje de que debe seleccionar antes una.} \\
			\multicolumn{3}{p{12cm}}{ - El usuario exporta una ROD seleccionada, elige el nombre, que por defecto es su nombre y se guarda correctamente en formato XML.} \\ \hline
		\end{tabular}
	\end{center}
	\caption{Historia de usuario - Exportar ROD}
	\label{tab:analisis/hu-exportar-rod}
\end{table}

\begin{table}[H]
	\begin{center}
		\begin{tabular} {l|c|l}
			\hline
			4.4. & \multicolumn{2}{c}{Importar ROD} \\ \noalign{\hrule height 1pt}
			\multicolumn{3}{l}{Descripción} \\ \hline
			\multicolumn{3}{p{12cm}}{Se debe poder leer un archivo XML con un formato específico con información de una ROD y sus componentes. Para ello se le mostrará al usuario un cuadro de diálogo donde explorará entre sus archivos, realizando un filtrado para mostrar solo los ficheros XML} \\ \noalign{\hrule height 1pt}
			\multicolumn{2}{l|}{Estimación} & 3 \\ \hline
			\multicolumn{2}{l|}{Prioridad} & 2 \\ \hline
			\multicolumn{2}{l|}{Dependencias} & 4.3. \\ \noalign{\hrule height 1pt}
			\multicolumn{3}{l}{Pruebas de aceptación} \\ \hline
			\multicolumn{3}{p{12cm}}{ - El usuario carga un archivo con el formato válido y se importa la ROD con todos sus componentes.} \\ \hline
		\end{tabular}
	\end{center}
	\caption{Historia de usuario - Importar ROD}
	\label{tab:analisis/hu-importar-rod}
\end{table}

\begin{table}[H]
	\begin{center}
		\begin{tabular} {l|c|l}
			\hline
			4.5. & \multicolumn{2}{c}{Cambiar ROD} \\ \noalign{\hrule height 1pt}
			\multicolumn{3}{l}{Descripción} \\ \hline
			\multicolumn{3}{p{12cm}}{El usuario debe poder cambiar la ROD activa seleccionándola dentro de la lista de ROD. Esto cambiará automáticamente la posición del plano para usar la de esta ROD.} \\ \noalign{\hrule height 1pt}
			\multicolumn{2}{l|}{Estimación} & 3 \\ \hline
			\multicolumn{2}{l|}{Prioridad} & 1 \\ \hline
			\multicolumn{2}{l|}{Dependencias} & 4.1. \\ \noalign{\hrule height 1pt}
			\multicolumn{3}{l}{Pruebas de aceptación} \\ \hline
			\multicolumn{3}{p{12cm}}{ - El usuario selecciona la misma ROD que está activa y no cambia.} \\
			\multicolumn{3}{p{12cm}}{ - El usuario selecciona una ROD que no está activa y cambia la posición del plano y se muestran los elementos de esa ROD.} \\ \hline
		\end{tabular}
	\end{center}
	\caption{Historia de usuario - Cambiar ROD}
\label{tab:analisis/hu-cambiar-rod}
\end{table}

\begin{table}[H]
	\begin{center}
		\begin{tabular} {l|c|l}
			\hline
			4.6. & \multicolumn{2}{c}{Crear regla} \\ \noalign{\hrule height 1pt}
			\multicolumn{3}{l}{Descripción} \\ \hline
			\multicolumn{3}{p{12cm}}{El sistema debe proveer de una funcionalidad para poder realizar medidas a tamaño real de zonas en un corte. Para ello el usuario solo tendrá que seleccionar dos puntos y le aparecerá la medida en milímetros. Se hará uso de un \textit{widget} de VTK para ello.} \\ \noalign{\hrule height 1pt}
			\multicolumn{2}{l|}{Estimación} & 3 \\ \hline
			\multicolumn{2}{l|}{Prioridad} & 2 \\ \hline
			\multicolumn{2}{l|}{Dependencias} & - \\ \noalign{\hrule height 1pt}
			\multicolumn{3}{l}{Pruebas de aceptación} \\ \hline
			\multicolumn{3}{p{12cm}}{ - El usuario selecciona dos puntos y le aparecerá la medida entre ambos puntos.} \\ \hline
		\end{tabular}
	\end{center}
	\caption{Historia de usuario - Crear regla}
	\label{tab:analisis/hu-crear-regla}
\end{table}

\begin{table}[H]
	\begin{center}
		\begin{tabular} {l|c|l}
			\hline
			4.7. & \multicolumn{2}{c}{Eliminar regla} \\ \noalign{\hrule height 1pt}
			\multicolumn{3}{l}{Descripción} \\ \hline
			\multicolumn{3}{p{12cm}}{Se deben poder eliminar medidas de distancias de la pantalla. Para ello el usuario solo tiene que seleccionarla y pulsar en eliminar.} \\ \noalign{\hrule height 1pt}
			\multicolumn{2}{l|}{Estimación} & 3 \\ \hline
			\multicolumn{2}{l|}{Prioridad} & 4 \\ \hline
			\multicolumn{2}{l|}{Dependencias} & 4.6. \\ \noalign{\hrule height 1pt}
			\multicolumn{3}{l}{Pruebas de aceptación} \\ \hline
			\multicolumn{3}{p{12cm}}{ - El usuario realiza la acción de eliminar una regla sin que haya ninguna y se le mostrará un mensaje indicando que antes seleccione una.} \\
			\multicolumn{3}{p{12cm}}{ - El usuario selecciona una regla, la elimina y desaparece de la lista y de la vista.} \\ \hline
		\end{tabular}
	\end{center}
	\caption{Historia de usuario - Eliminar regla}
	\label{tab:analisis/hu-eliminar-regla}
\end{table}

\begin{table}[H]
	\begin{center}
		\begin{tabular} {l|c|l}
			\hline
			4.8. & \multicolumn{2}{c}{Editar regla} \\ \noalign{\hrule height 1pt}
			\multicolumn{3}{l}{Descripción} \\ \hline
			\multicolumn{3}{p{12cm}}{El usuario debe ser capaz de editar una medida por si se ha equivocado. Para ello selecciona y arrastra uno de los puntos de ésta.} \\ \noalign{\hrule height 1pt}
			\multicolumn{2}{l|}{Estimación} & 1 \\ \hline
			\multicolumn{2}{l|}{Prioridad} & 5 \\ \hline
			\multicolumn{2}{l|}{Dependencias} & 4.6. \\ \noalign{\hrule height 1pt}
			\multicolumn{3}{l}{Pruebas de aceptación} \\ \hline
			\multicolumn{3}{p{12cm}}{ - El usuario selecciona y arrastra un punto y se actualiza la posición de éste y el valor de la medida de la distancia entre ambos.} \\ \hline
		\end{tabular}
	\end{center}
	\caption{Historia de usuario - Editar regla}
	\label{tab:analisis/hu-editar-regla}
\end{table}

\begin{table}[H]
	\begin{center}
		\begin{tabular} {l|c|l}
			\hline
			4.9. & \multicolumn{2}{c}{Ocultar regla} \\ \noalign{\hrule height 1pt}
			\multicolumn{3}{l}{Descripción} \\ \hline
			\multicolumn{3}{p{12cm}}{El usuario debe ser capaz de ocultar una regla de la vista sin eliminarla. Para ello la selecciona en la lista y pulsa en ocultar. Se mostrará de forma distinta en la lista de reglas para que se tenga claro que se ha ocultado.} \\ \noalign{\hrule height 1pt}
			\multicolumn{2}{l|}{Estimación} & 2 \\ \hline
			\multicolumn{2}{l|}{Prioridad} & 6 \\ \hline
			\multicolumn{2}{l|}{Dependencias} & 4.6. \\ \noalign{\hrule height 1pt}
			\multicolumn{3}{l}{Pruebas de aceptación} \\ \hline
			\multicolumn{3}{p{12cm}}{ - El usuario realiza la acción de ocultar una regla sin que haya ninguna y se le mostrará un mensaje indicando que antes seleccione una.} \\
			\multicolumn{3}{p{12cm}}{ - El usuario selecciona una regla, la oculta y desaparece de la vista, pero no de la lista.} \\ \hline
		\end{tabular}
	\end{center}
	\caption{Historia de usuario - Ocultar regla}
	\label{tab:analisis/hu-ocultar-regla}
\end{table}

\begin{table}[H]
	\begin{center}
		\begin{tabular} {l|c|l}
			\hline
			4.10. & \multicolumn{2}{c}{Mostrar regla} \\ \noalign{\hrule height 1pt}
			\multicolumn{3}{l}{Descripción} \\ \hline
			\multicolumn{3}{p{12cm}}{El usuario debe ser capaz de mostrar una regla que previamente ha ocultado. Para ello la selecciona en la lista y pulsa en mostrar. Se volverá a mostrar como antes en la lista de reglas.} \\ \noalign{\hrule height 1pt}
			\multicolumn{2}{l|}{Estimación} & 1 \\ \hline
			\multicolumn{2}{l|}{Prioridad} & 6 \\ \hline
			\multicolumn{2}{l|}{Dependencias} & 4.9. \\ \noalign{\hrule height 1pt}
			\multicolumn{3}{l}{Pruebas de aceptación} \\ \hline
			\multicolumn{3}{p{12cm}}{ - El usuario realiza la acción de mostrar una regla sin que haya ninguna y se le mostrará un mensaje indicando que antes seleccione una.} \\
			\multicolumn{3}{p{12cm}}{ - El usuario selecciona una regla, la muestra, vuelve a aparecer en la vista y se muestra como antes en la lista.} \\ \hline
		\end{tabular}
	\end{center}
	\caption{Historia de usuario - Mostrar regla}
	\label{tab:analisis/hu-mostrar-regla}
\end{table}

\begin{table}[H]
	\begin{center}
		\begin{tabular} {l|c|l}
			\hline
			4.11. & \multicolumn{2}{c}{Crear transportador de ángulos} \\ \noalign{\hrule height 1pt}
			\multicolumn{3}{l}{Descripción} \\ \hline
			\multicolumn{3}{p{12cm}}{El sistema debe proveer de una funcionalidad para poder realizar medidas de ángulos a tamaño real de zonas en un corte. Para ello el usuario solo tendrá que seleccionar tres puntos y le aparecerá la medida del ángulo en grados. Se hará uso de un \textit{widget} de VTK para ello.} \\ \noalign{\hrule height 1pt}
			\multicolumn{2}{l|}{Estimación} & 3 \\ \hline
			\multicolumn{2}{l|}{Prioridad} & 3 \\ \hline
			\multicolumn{2}{l|}{Dependencias} & - \\ \noalign{\hrule height 1pt}
			\multicolumn{3}{l}{Pruebas de aceptación} \\ \hline
			\multicolumn{3}{p{12cm}}{ - El usuario selecciona tres puntos y le aparecerá la medida del ángulo entre esos puntos.} \\ \hline
		\end{tabular}
	\end{center}
	\caption{Historia de usuario - Crear transportador de ángulos}
	\label{tab:analisis/hu-crear-transportador-angulos}
\end{table}

\begin{table}[H]
	\begin{center}
		\begin{tabular} {l|c|l}
			\hline
			4.12. & \multicolumn{2}{c}{Eliminar transportador de ángulos} \\ \noalign{\hrule height 1pt}
			\multicolumn{3}{l}{Descripción} \\ \hline
			\multicolumn{3}{p{12cm}}{Se deben poder eliminar medidas de ángulos de la pantalla. Para ello el usuario solo tiene que seleccionarlo y pulsar en eliminar.} \\ \noalign{\hrule height 1pt}
			\multicolumn{2}{l|}{Estimación} & 3 \\ \hline
			\multicolumn{2}{l|}{Prioridad} & 4 \\ \hline
			\multicolumn{2}{l|}{Dependencias} & 4.11. \\ \noalign{\hrule height 1pt}
			\multicolumn{3}{l}{Pruebas de aceptación} \\ \hline
			\multicolumn{3}{p{12cm}}{ - El usuario realiza la acción de eliminar un transportador de ángulos sin que haya ninguno y se le mostrará un mensaje indicando que antes seleccione uno.} \\
			\multicolumn{3}{p{12cm}}{ - El usuario selecciona una regla, la elimina y desaparece de la lista y de la vista.} \\ \hline
		\end{tabular}
	\end{center}
	\caption{Historia de usuario - Eliminar transportador de ángulos}
	\label{tab:analisis/hu-eliminar-transportador-angulos}
\end{table}

\begin{table}[H]
	\begin{center}
		\begin{tabular} {l|c|l}
			\hline
			4.13. & \multicolumn{2}{c}{Editar transportador de ángulos} \\ \noalign{\hrule height 1pt}
			\multicolumn{3}{l}{Descripción} \\ \hline
			\multicolumn{3}{p{12cm}}{El usuario debe ser capaz de editar una medida de ángulos por si se ha equivocado. Para ello selecciona y arrastra uno de los puntos de éste.} \\ \noalign{\hrule height 1pt}
			\multicolumn{2}{l|}{Estimación} & 1 \\ \hline
			\multicolumn{2}{l|}{Prioridad} & 5 \\ \hline
			\multicolumn{2}{l|}{Dependencias} & 4.11. \\ \noalign{\hrule height 1pt}
			\multicolumn{3}{l}{Pruebas de aceptación} \\ \hline
			\multicolumn{3}{p{12cm}}{ - El usuario selecciona y arrastra un punto y se actualiza la posición de éste y el valor de la medida del ángulo entre ambos.} \\ \hline
		\end{tabular}
	\end{center}
	\caption{Historia de usuario - Editar transportador de ángulos}
	\label{tab:analisis/hu-editar-transportador-angulos}
\end{table}

\begin{table}[H]
	\begin{center}
		\begin{tabular} {l|c|l}
			\hline
			4.14. & \multicolumn{2}{c}{Ocultar transportador de ángulos} \\ \noalign{\hrule height 1pt}
			\multicolumn{3}{l}{Descripción} \\ \hline
			\multicolumn{3}{p{12cm}}{El usuario debe ser capaz de ocultar un transportador de ángulos de la vista sin eliminarlo. Para ello lo selecciona en la lista y pulsa en ocultar. Se mostrará de forma distinta en la lista de transportadores de ángulos para que se tenga claro que se ha ocultado.} \\ \noalign{\hrule height 1pt}
			\multicolumn{2}{l|}{Estimación} & 2 \\ \hline
			\multicolumn{2}{l|}{Prioridad} & 6 \\ \hline
			\multicolumn{2}{l|}{Dependencias} & 4.11. \\ \noalign{\hrule height 1pt}
			\multicolumn{3}{l}{Pruebas de aceptación} \\ \hline
			\multicolumn{3}{p{12cm}}{ - El usuario realiza la acción de ocultar un transportador de ángulos sin que haya ninguno y se le mostrará un mensaje indicando que antes seleccione uno.} \\
			\multicolumn{3}{p{12cm}}{ - El usuario selecciona un transportador de ángulos, lo oculta y desaparece de la vista, pero no de la lista.} \\ \hline
		\end{tabular}
	\end{center}
	\caption{Historia de usuario - Ocultar transportador de ángulos}
	\label{tab:analisis/hu-ocultar-transportador-angulos}
\end{table}

\begin{table}[H]
	\begin{center}
		\begin{tabular} {l|c|l}
			\hline
			4.15. & \multicolumn{2}{c}{Mostrar transportador de ángulos} \\ \noalign{\hrule height 1pt}
			\multicolumn{3}{l}{Descripción} \\ \hline
			\multicolumn{3}{p{12cm}}{El usuario debe ser capaz de mostrar un transportador de ángulos que previamente ha ocultado. Para ello lo selecciona en la lista y pulsa en mostrar. Se volverá a mostrar como antes en la lista de transportadores de ángulos.} \\ \noalign{\hrule height 1pt}
			\multicolumn{2}{l|}{Estimación} & 1 \\ \hline
			\multicolumn{2}{l|}{Prioridad} & 7 \\ \hline
			\multicolumn{2}{l|}{Dependencias} & 4.14. \\ \noalign{\hrule height 1pt}
			\multicolumn{3}{l}{Pruebas de aceptación} \\ \hline
			\multicolumn{3}{p{12cm}}{ - El usuario realiza la acción de mostrar un transportador de ángulos sin que haya ninguno y se le mostrará un mensaje indicando que antes seleccione uno.} \\
			\multicolumn{3}{p{12cm}}{ - El usuario selecciona un transportador de ángulos, lo muestra, vuelve a aparecer en la vista y se muestra como antes en la lista.} \\ \hline
		\end{tabular}
	\end{center}
	\caption{Historia de usuario - Mostrar transportador de ángulos}
	\label{tab:analisis/hu-mostrar-transportador-angulos}
\end{table}

\begin{table}[H]
	\begin{center}
		\begin{tabular} {l|c|l}
			\hline
			4.16. & \multicolumn{2}{c}{Crear nota} \\ \noalign{\hrule height 1pt}
			\multicolumn{3}{l}{Descripción} \\ \hline
			\multicolumn{3}{p{12cm}}{.} \\ \noalign{\hrule height 1pt}
			\multicolumn{2}{l|}{Estimación} & 4 \\ \hline
			\multicolumn{2}{l|}{Prioridad} & 2 \\ \hline
			\multicolumn{2}{l|}{Dependencias} & - \\ \noalign{\hrule height 1pt}
			\multicolumn{3}{l}{Pruebas de aceptación} \\ \hline
			\multicolumn{3}{p{12cm}}{ - .} \\ \hline
		\end{tabular}
	\end{center}
	\caption{Historia de usuario - Crear nota}
	\label{tab:analisis/hu-crear-nota}
\end{table}

\begin{table}[H]
	\begin{center}
		\begin{tabular} {l|c|l}
			\hline
			4.17. & \multicolumn{2}{c}{Eliminar nota} \\ \noalign{\hrule height 1pt}
			\multicolumn{3}{l}{Descripción} \\ \hline
			\multicolumn{3}{p{12cm}}{.} \\ \noalign{\hrule height 1pt}
			\multicolumn{2}{l|}{Estimación} & 2 \\ \hline
			\multicolumn{2}{l|}{Prioridad} & 4 \\ \hline
			\multicolumn{2}{l|}{Dependencias} & 4.16. \\ \noalign{\hrule height 1pt}
			\multicolumn{3}{l}{Pruebas de aceptación} \\ \hline
			\multicolumn{3}{p{12cm}}{ - .} \\ \hline
		\end{tabular}
	\end{center}
	\caption{Historia de usuario - Eliminar nota}
	\label{tab:analisis/hu-eliminar-nota}
\end{table}

\begin{table}[H]
	\begin{center}
		\begin{tabular} {l|c|l}
			\hline
			4.18. & \multicolumn{2}{c}{Editar nota} \\ \noalign{\hrule height 1pt}
			\multicolumn{3}{l}{Descripción} \\ \hline
			\multicolumn{3}{p{12cm}}{.} \\ \noalign{\hrule height 1pt}
			\multicolumn{2}{l|}{Estimación} & 3 \\ \hline
			\multicolumn{2}{l|}{Prioridad} & 5 \\ \hline
			\multicolumn{2}{l|}{Dependencias} & 4.16. \\ \noalign{\hrule height 1pt}
			\multicolumn{3}{l}{Pruebas de aceptación} \\ \hline
			\multicolumn{3}{p{12cm}}{ - .} \\ \hline
		\end{tabular}
	\end{center}
	\caption{Historia de usuario - Editar nota}
	\label{tab:analisis/hu-editar-nota}
\end{table}

\begin{table}[H]
	\begin{center}
		\begin{tabular} {l|c|l}
			\hline
			4.19. & \multicolumn{2}{c}{Ocultar nota} \\ \noalign{\hrule height 1pt}
			\multicolumn{3}{l}{Descripción} \\ \hline
			\multicolumn{3}{p{12cm}}{.} \\ \noalign{\hrule height 1pt}
			\multicolumn{2}{l|}{Estimación} & 2 \\ \hline
			\multicolumn{2}{l|}{Prioridad} & 6 \\ \hline
			\multicolumn{2}{l|}{Dependencias} & 4.16. \\ \noalign{\hrule height 1pt}
			\multicolumn{3}{l}{Pruebas de aceptación} \\ \hline
			\multicolumn{3}{p{12cm}}{ - .} \\ \hline
		\end{tabular}
	\end{center}
	\caption{Historia de usuario - Ocultar nota}
	\label{tab:analisis/hu-ocultar-nota}
\end{table}

\begin{table}[H]
	\begin{center}
		\begin{tabular} {l|c|l}
			\hline
			4.20. & \multicolumn{2}{c}{Mostrar nota} \\ \noalign{\hrule height 1pt}
			\multicolumn{3}{l}{Descripción} \\ \hline
			\multicolumn{3}{p{12cm}}{.} \\ \noalign{\hrule height 1pt}
			\multicolumn{2}{l|}{Estimación} & 1 \\ \hline
			\multicolumn{2}{l|}{Prioridad} & 6 \\ \hline
			\multicolumn{2}{l|}{Dependencias} & 4.19. \\ \noalign{\hrule height 1pt}
			\multicolumn{3}{l}{Pruebas de aceptación} \\ \hline
			\multicolumn{3}{p{12cm}}{ - .} \\ \hline
		\end{tabular}
	\end{center}
	\caption{Historia de usuario - Mostrar nota}
	\label{tab:analisis/hu-mostrar-nota}
\end{table}

\subsubsection{Mejoras en código}

\begin{table}[H]
	\begin{center}
		\begin{tabular} {l|c|l}
			\hline
			5.1. & \multicolumn{2}{c}{Internacionalización} \\ \noalign{\hrule height 1pt}
			\multicolumn{3}{l}{Descripción} \\ \hline
			\multicolumn{3}{p{12cm}}{El usuario tiene que tener la posibilidad de utilizar la aplicación en dos idiomas distintos, español e inglés. Para ello se utilizará el mecanismo de internacionalización que provee Qt y se generarán dos instaladores: uno en español y otro en inglés.} \\ \noalign{\hrule height 1pt}
			\multicolumn{2}{l|}{Estimación} & 6 \\ \hline
			\multicolumn{2}{l|}{Prioridad} & 8 \\ \hline
			\multicolumn{2}{l|}{Dependencias} & - \\ \noalign{\hrule height 1pt}
			\multicolumn{3}{l}{Pruebas de aceptación} \\ \hline
			\multicolumn{3}{p{12cm}}{ - El usuario ejecutará el programa en cualquiera de los idiomas y se mostrarán los textos con el idioma correspondiente.} \\ \hline
		\end{tabular}
	\end{center}
	\caption{Historia de usuario - Internacionalización}
	\label{tab:analisis/hu-internacionalizacion}
\end{table}
\chapter{Diseño}

En este capítulo se mostrarán los diagramas arquitectónico, de paquetes y de clases de la aplicación. 

Estos diagramas UML han sido generados con la aplicación StarUML \cite{staruml} y están disponibles en la rama de diseño del repositorio del proyecto alojado en GitHub: \url{https://github.com/fblupi/3DCurator/tree/design} por si es necesario verlos a mayor tamaño.

\section{Arquitectura del software}

El software está íntegramente escrito en C++ a excepción de las interfaces gráficas que usan un formato específico basado en XML.

Se hace uso a nivel hardware tanto de la CPU como de la GPU usando OpenGL como librería gráfica de bajo nivel pese a no programar directamente con esta librería. En su lugar se utiliza la librería gráfica de alto nivel VTK que abstrae de la complejidad de OpenGL.

Para gestionar la interfaz de usuario se utiliza Qt que permite una buena integración con VTK.

Se usan también, como librerías adicionales, ITK para aplicar algoritmos de filtros a los volúmenes, OpenCV para los algoritmos de visión por computador y Boost para la gestión de ficheros XML.

Para generar el proyecto pre-compilando todas estas librerías citadas anteriormente, así como para hacerlo multiplataforma y poder compilarlo en cualquier sistema operativo se usa CMake.

\begin{table}[H]
	\begin{center}
		\begin{tabular}{|l|c|c|c|c|c|}
			\hline
			Librerías de alto nivel  & Boost        & VTK       & ITK    & OpenCV   & Qt   \\ \hline
			Librerías de bajo nivel  & \multicolumn{5}{l|}{OpenGL}                         \\ \hline
			Lenguaje de programación & \multicolumn{5}{l|}{C++}                            \\ \hline
			Nivel Hardware           & \multicolumn{2}{l|}{CPU} & \multicolumn{3}{l|}{GPU} \\ \hline
		\end{tabular}
	\end{center}
	\caption{Arquitectura del software}
	\label{tab:diseno/diagrama-arquitectonico}
\end{table}

\section{Diagrama de paquetes}

En el siguiente diagrama (Figura \ref{fig:diseno/package}) se muestran las dependencias entre los distintos paquetes cuyas clases se detallarán en el siguiente apartado.

\begin{figure}[H]
	\centering
	\includegraphics[width=12cm]{imagenes/diseno/package}
	\caption{Diagrama de paquetes de 3DCurator}
	\label{fig:diseno/package}
\end{figure}

\section{Diagramas de clases}

Se presentan las distintas clases de los distintos módulos que contiene 3DCurator.

\subsection{\textit{Chart}}

Este módulo contiene las clases auxiliares para controlar las gráficas utilizadas para visualizar y editar la función de transferencia:

\begin{figure}[H]
	\centering
	\includegraphics[width=12cm]{imagenes/diseno/chart}
	\caption{Diagrama de clases del paquete \textit{Chart}}
	\label{fig:diseno/chart}
\end{figure}

\subsection{\textit{Core}}

Este es el módulo principal de 3DCurator. Contiene las clases que gestionan los datos del volumen (\textit{Sculpture}), la función de transferencia (\textit{Transfer Function}) y plano de corte (\textit{SlicePlane}):

\begin{figure}[H]
	\centering
	\includegraphics[width=12cm]{imagenes/diseno/core}
	\caption{Diagrama de clases del paquete \textit{Core}}
	\label{fig:diseno/core}
\end{figure}

\subsection{\textit{Documentation}}

En este módulo se encuentra la clase que almacena los distintos elementos utilizados para la documentación:

\begin{figure}[H]
	\centering
	\includegraphics[width=12cm]{imagenes/diseno/documentation}
	\caption{Diagrama de clases del paquete \textit{Documentation}}
	\label{fig:diseno/documentation}
\end{figure}

\subsection{\textit{GUI}}

Este módulo contiene las distintas clases que gestionan las ventanas de la GUI:

\begin{figure}[H]
	\centering
	\includegraphics[width=12cm]{imagenes/diseno/gui}
	\caption{Diagrama de clases del paquete \textit{GUI}}
	\label{fig:diseno/gui}
\end{figure}

\subsection{\textit{Interactor}}

Este módulo contiene los distintos interactuadores con las ventanas de visualización: 

\begin{figure}[H]
	\centering
	\includegraphics[width=12cm]{imagenes/diseno/interactor}
	\caption{Diagrama de clases del paquete \textit{Interactor}}
	\label{fig:diseno/interactor}
\end{figure}

\subsection{\textit{Segmentation}}

En este módulo se encuentran las clases utilizadas para realizar segmentaciones en el volumen:

\begin{figure}[H]
	\centering
	\includegraphics[width=12cm]{imagenes/diseno/segmentation}
	\caption{Diagrama de clases del paquete \textit{Segmentation}}
	\label{fig:diseno/segmentation}
\end{figure}

\subsection{\textit{Util}}

Este módulo contiene funciones utilizadas frecuentemente por distintos módulos del programa:

\begin{figure}[H]
	\centering
	\includegraphics[width=12cm]{imagenes/diseno/util}
	\caption{Diagrama de clases del paquete \textit{Util}}
	\label{fig:diseno/util}
\end{figure}

\subsection{\textit{Widget}}

Este módulo contiene el \textit{widget} personalizado de VTK que se utiliza:

\begin{figure}[H]
	\centering
	\includegraphics[width=5cm]{imagenes/diseno/widget}
	\caption{Diagrama de clases del paquete \textit{Widget}}
	\label{fig:diseno/widget}
\end{figure}
\chapter{Desarrollo del trabajo}

En este capítulo se realizará un estudio detallado del estado del arte para posteriormente detallar las distintas fases de desarrollo y los problemas que han surgido con sus respectivas soluciones.

\section{Desarrollo profundo del estado del arte}

Este trabajo es la continuación de mi anteior TFG donde traté el tema de la renderización de conjuntos de datos volumétricos, este trabajo está dirigido al tratamiento de este conjunto de datos.

Un conjunto de datos volumétrico o campo escalar 3D está representado por una función $R^{3} \rightarrow R$. En otras palabras, se trata de un conjunto de datos representado por una matriz tridimensional, donde cada elemento de esta matriz se puede denominar vóxel y es importante tener claro que el valor de este no es un color sino un valor como tal que posteriormente se renderizará con un color y una transparencia según una función de transferencia. Es ahí donde erradica la separación conceptual entre modelado y visualización de un conjunto de datos volumétrico.

El flujo a la hora de representar un conjunto de datos volumétricos sería el siguiente:

\begin{itemize}
	\item Obtención de imágenes
	\item Filtrado
	\item Segmentación
	\item Visualización
\end{itemize}

Durante el TFG traté el último de los pasos y en este TFM comentaré las distintas opciones del primero y trabajaré en técnicas del segundo y el tercero aplicadas a esculturas de madera policromadas.

\subsection{Obtención de imágenes}

Existen diversas técnicas de obtención de imágenes volumétricas. En esta sección se describirán las dos técnicas más usadas actualmente y se realizará una comparación entre ellas. No se han incluido técnicas como el PET, SPECT o ecografía pues necesitan de contrastes que no se podrían aplicar en una escultura si se quiere preservar su estado.

\subsubsection{Tomografía Computarizada}

La Tomografía Computarizada (TC o CT en inglés) fue la primera de las técnicas que surgió para la obtención de datos volumétricos pero ha ido evolucionando hasta el día de hoy como se detalló en la introducción.

\begin{figure}[H]
	\centering
	\includegraphics[width=9cm]{imagenes/desarrollo/tc}
	\caption{Escáner TC sin la carcasa, por lo que se puede ver sus componentes internos. T: Tubo rayos X, D: Detectores rayos X, X: Haces de rayos X y R: \textit{Gantry}. Imagen extraída de: \url{https://en.wikipedia.org/wiki/File:Ct-internals.jpg}}
	\label{fig:desarrollo/tc}
\end{figure}

Los parámetros utilizados en una TC son los siguientes:

\begin{itemize}
	\item \textbf{Resolución espacial} (número de cortes, píxeles por corte y distancia entre vóxeles): Si la resolución es más alta los datos serán más ruidosos si la dosis de radiación se mantiene.
	\item \textbf{Dosis de radiación}: Si la dosis de radiación es mayor se conseguirá mejor ratio señal-ruido y las imágenes podrán ser de mayor resolución sin que el ruido sea un problema.
	\item \textbf{\textit{Gantry tilt}} (sistema de rotación emisor/receptor): Se puede adaptar el \textit{gantry tilt} para una parte específica a examinar.
\end{itemize}

Los valores de intensidad de las imágenes extraídas se encuentran en unidades Hounsfield (HU). Que es una unidad de medida normalizada que hace que un tejido tenga ese valor sean cuales sean los parámetros del escáner.

\subsubsection{Imagen por Resonancia Magnética}

La Imagen por Resonancia Magnética (IRM o MRI en inglés) es una técnica en la que, a diferencia de la TC donde se usa radiación ionizante, se usan campos mágnéticos para diferenciar los distintos materiales del objeto escaneado, específicamente la ocurrencia del núcleo de hidrógeno que son capaces de absorber y emitir radio frecuencia cuando se colocan en un campo magnético externo \cite{mcrobbie10}.

\begin{figure}[H]
	\centering
	\includegraphics[width=12cm]{imagenes/desarrollo/irm-longitudinal}
	\caption{Esquema de un escáner IRM (Sección Longitudinal). Imagen extraída de: \url{https://en.wikipedia.org/wiki/File:Mri_scanner_schematic_labelled.svg}}
	\label{fig:desarrollo/irm-longitudinal}
\end{figure}

\begin{figure}[H]
	\centering
	\includegraphics[width=12cm]{imagenes/desarrollo/irm-axial}
	\caption{Esquema de un escáner IRM (Sección Axial). Imagen extraída de: \url{https://en.wikipedia.org/wiki/File:Mri_scanner_schematic_labelled.svg}}
	\label{fig:desarrollo/irm-axial}
\end{figure}

El campo magnético alinea los momentos magnéticos de los núcleos atómicos de hidrógeno en dirección paralela y anti-paralela. 

A continuación se emite radiación electromagnética a un pulso de radiofrecuencia determinado. Algunos núcleos que se encuentran en dirección paralela pasarán a anti-paralela y al volver a su dirección original perderán energía en forma de fotones que podrán ser detectados. 

Estos dos tiempos, T1 (\textit{phase}) y T2 (\textit{dephase}) son medidos. Para obtener T1 hay que ver el valor en la gráfica de relajación longitudinal al 63\% y para obtener T2 hay que hacer lo mismo en la gráfica de relajación transversal al 37\% \cite{relaxation}.

\begin{figure}[H]
	\centering
	\includegraphics[width=12cm]{imagenes/desarrollo/relajacion-longitudinal-transversal}
	\caption{A la izquierda gráfica de la relajación longitudinal (crecimiento logarítmico) y a la derecha gráfica de la relajación transversal (crecimiento exponencial) \cite{relaxation}}
	\label{fig:desarrollo/relajacion-longitudinal-transversal}
\end{figure}

\subsubsection{Comparación entre TC e IRM}

Las principales diferencias entre ambas técnicas son las siguientes:

\begin{itemize}
	\item La IRM obtiene imágenes de menor resolución que la TC.
	\item La IRM proporciona un mayor contraste entre tejidos poco densos.
	\item Los datos obtenidos con una TC son más entendibles por médicos mientras que los obtenidos con una IRM por radiólogos.
	\item El tiempo y coste de escaneo de una IRM es mayor que el de una TC.
	\item Los datos obtenidos con una TC se encuentran en unidades normalizadas mientras que los obtenidos con una IRM variarán dependiendo de los parámetros del escáner.
\end{itemize}

Al necesitar unos \textit{presets} con los que se pueda visualizar la escultura sin necesidad de que el usuario edite la función de transferencia, el último punto nos haría decantar por la TC. Además gracias a esta podremos obtener imágenes de mayor resolución. El único punto en contra en esta elección es que el contraste entre materiales de baja densidad (como la madera) no será tan distinguible como con la IRM.

\subsection{Filtrado}

Los datos en crudo obtenidos con las técnicas anteriormente descritas muchas veces no son lo suficientemente buenas y tienen lo que se denomina ruido. Hay muchos tipos de ruido y existen distintos filtros que aplicar a las imágenes para reducirlo.

Antes de describir los filtros más usados, se va a profundizar en ciertos aspectos teóricos necesarios para entender mejor cómo funcionan. Estos conceptos se van a describir para un espacio 2D, aunque pasar a un espacio 3D como el de los volúmenes es trivial pues tan solo haría falta utilizar una variable más para la profundidad.

Lo primero que hay que comprender es el concepto de vecindario. Que no es más que los píxeles que lo rodean a una distancia concreta. Por ejemplo un vecindario de tamaño 3x3 sobre el punto $p$ es un conjunto de píxeles con tamaño 3x3 con centro en el píxel $p$ (Figura \ref{fig:desarrollo/vecindario}).

\begin{figure}[H]
	\centering
	\includegraphics[width=10cm]{imagenes/desarrollo/vecindario}
	\caption{Vecindarios 3x3, 5x5 y 7x7 de un píxel (rojo).}
	\label{fig:desarrollo/vecindario}
\end{figure}

Un dominio espacial se denota con la expresión:

\[ g(x, y) = T[f(x, y)] \]

Donde:

\begin{itemize}
	\item $f(x, y)$ es la imagen de entrada
	\item $g(x, y)$ es la imagen de salida
	\item $T$ es un operador en $f$ definido sobre el vecindario $(x,y)$
\end{itemize}

Los filtros espaciales consisten en aplicar el operador $T$ a los píxeles del vecindario. Por ejemplo en un vecindario 3x3 y un operador $T$ definido como la intensidad media del vecindario, el valor $g(x_{i}, y_{j})$ será la suma del valor $f(x_{i}, y_{j})$ de su vecindario dividido entre 9.

\subsubsection{Filtro media}

El filtro media es el definido anteriormente. Es decir, usa un \textit{kernel} en el que todos los vecinos tienen el mismo peso.

Es un filtro utilizado para suavizar imágenes con mucho ruido.

\subsubsection{Filtro mediana}

El filtro media hace uso de la mediana para calcular el valor de salida del píxel. Por ejemplo, para la matriz: 

\[
\begin{bmatrix}
	1 & 4 & 0 \\
	2 & 2 & 4 \\
	1 & 0 & 1 
\end{bmatrix} 
\]

El valor para el punto $p$ correspondiente a $M(1, 1)$ con valor original 2, sería 1. Porque es la mediana de su vecindario (0, 0, 1, 1, \textbf{1}, 2, 2, 4, 4).

Este filtro es muy usado por ser el más efectivo para reducir el ruido de tipo \textit{salt-and-pepper} 

\begin{figure}[H]
	\centering
	\includegraphics[width=11cm]{imagenes/desarrollo/salt-and-pepper}
	\caption{Ejemplo de ruido tipo \textit{salt-and-pepper}, utilizando un filtro mediana.  Imagen extraída de: \url{https://en.wikipedia.org/wiki/File:Medianfilterp.png}}
	\label{fig:desarrollo/salt-and-pepper}
\end{figure}

\subsubsection{Filtro \textit{gaussiano}}

El filtro \textit{gaussiano} o binomial (Figura \ref{fig:desarrollo/filtro-gaussiano}) hace uso de una versión discretizada de la función \textit{gaussiana} y, por tanto, se basa en la convolución de una matriz que para un vecindario 5x5 sería:

\[
\begin{bmatrix}
	1 & 4 & 6 & 4 & 1 \\
	4 & 16 & 24 & 16 & 4 \\
	6 & 24 & 36 & 24 & 6 \\
	4 & 16 & 24 & 16 & 4 \\
	1 & 4 & 6 & 4 & 1
\end{bmatrix} 
\]

Para normalizar los elementos del \textit{kernel} haría falta que sumasen 1, por lo que se divide entre la suma de todos sus elementos (256 en el caso de 5x5).

\begin{figure}[H]
	\centering
	\includegraphics[width=11cm]{imagenes/desarrollo/filtro-gaussiano}
	\caption{Ejemplo de suavizado usado un fitro \textit{gaussiano}}
	\label{fig:desarrollo/filtro-gaussiano}
\end{figure}

Este filtro, al igual que el que hace uso de la media es, es un filtro paso baja utilizado para suavizar imágenes.

\subsection{Segmentación}

La segmentación es la separación de partes de un volumen para su estudio por separado. Antes de profundizar en algunas de las técnicas utilizadas en la actualidad, hay que tener claro una serie de conceptos \cite{segmentation-concepts}.

\begin{itemize}
	\item \textbf{Adyacencia}: Dos píxel son adyacentes si son vecinos y satisfacen un criterio de similitud, por ejemplo, que su valor de intensidad esté entre -700 y -300.
	\item \textbf{Camino}: Un camino entre dos píxeles es una secuencia de píxeles adyacentes que va desde un píxel hasta otro.
	\item \textbf{Conectividad}: Existe conectividad entre dos píxeles si se puede trazar al menos un camino entre ambos.
	\item \textbf{Componente conectado en un subconjunto de una imagen}: En un subconjunto $S$ de una imagen, para todos los píxeles $p$ en $S$, el conjunto de píxeles en $S$ conectados a $p$ se denominan componentes conectados en $S$.
	\item \textbf{Conjunto conectado}: Si el subconjunto de una imagen $S$ solo tiene un componente conectado, entonces se convierte en un conjunto conectado.
	\item \textbf{Región}: Un subconjunto $S$ de una imagen es una región si es un conjunto conectado.
	\item \textbf{Borde}: El borde de una región $S$ es el conjunto de píxeles que tienen uno o más vecinos no pertenecientes a $S$.
\end{itemize}

La segmentación trata de buscar los vóxeles que pertenecen a una estructura. Existen distintas aproximaciones como se detallaron en la introducción. A continuación se detallarán algunos de los métodos que se citaron.

\subsubsection{Segmentación basada en umbrales}

Esta segmentación es muy básica y trata de diferenciar las regiones utilizando dos umbrales de valor de intensidad y realizando un filtro en el que se descartasen todos aquellos vóxeles que no se encuentran entre estos valores \cite{otsu79}.

El uso de los histogramas de valores de la imagen pueden resultar útiles a la hora de seleccionar los umbrales.

Este método puede resultar útil para separar materiales con diferencias notables entre valores de densidad. En nuestro caso podría servir para separar madera de estuco, por ejemplo.

\subsubsection{Segmentación basada en crecimiento usando umbrales}

La segmentación basada en crecimiento usando umbrales es una extensión del método anterior, con la diferencia de que la segmentación basada en umbrales obtiene todas las regiones que se encuentran en la imagen y en la basada en crecimiento solo obtiene una región, obteniendo todos los puntos conectados a uno inicial que tienen el umbral como criterio de similitud \cite{haralick85} (Figura \ref{fig:desarrollo/segmentacion-crecimiento-umbral}).

\begin{figure}[H]
	\centering
	\includegraphics[width=11cm]{imagenes/desarrollo/segmentacion-crecimiento-umbral}
	\caption{Segmentación de los plexos coirodeos de una imagen de un cerebro usando una segmentación basada en crecimiento usando un umbral entre 210 y 255}
	\label{fig:desarrollo/segmentacion-crecimiento-umbral}
\end{figure}

Una variante de este método sería usar umbrales dinámicos. Es decir, en lugar de utilizar un valor máximo y mínimo global para toda la imagen (por ejemplo, valores de densidad entre -700 y -300), usar un rango dependiendo del punto donde se encuentro (por ejemplo, si el rango es de 100 y estamos en un punto con valor -532, el umbral para sus vecinos estaría entre -632 y -432). Este es el método de segmentación que se implementó en el TFG para segmentar las piezas separadas de la escultura como los elementos de la camilla.

En este caso es muy importante elegir el punto inicial, pues si se escoge un punto que se encuentre cercano al borde, puede hacer que la segmentación se extienda a zonas no deseadas.

\subsubsection{\textit{Watershed}}

La transformación divisoria, más conocida como \textit{watershed} se basa en ver las imágenes como un relieve topográfico con crestas y cuencas. Las elevaciones del terreno estarían definidas por los valores de densidad de la imagen o por el gradiente \cite{beucher79} (Figura \ref{fig:desarrollo/watershed}). 

El resultado es la descomposición de la imagen en distintas cuencas hidrográficas para cada mínimo local. Pero el elevado número de mínimos locales da lugar a sobresegmentación por lo que hay que definir un método de mezclado. El método más utilizado es el de la inundación. Se marca una posición de inundación que podría mezclar varias cuencas.

\begin{figure}[H]
	\centering
	\includegraphics[width=9cm]{imagenes/desarrollo/watershed}
	\caption{El terreno se indica con una línea continua negra, las distintas regiones están marcadas con líneas discontinuas negras, las flechas los mínimos locales, las líneas discontinuas azules los niveles de agua y las líneas discontinuas azules los distintos niveles de agua para realizar la inundación. Imagen extraída de \url{https://en.wikipedia.org/wiki/File:Watershed_transform_-_flood_interpretation.svg}}
	\label{fig:desarrollo/watershed}
\end{figure}

\subsubsection{\textit{Livewire}}

El método de segmentación \textit{livewire} (también conocido como tijeras inteligentes) es, a diferencia de los anteriores, un método basado en bordes.

\textit{Livewire} es un método manual de segmentación donde hace falta la intervención del usuario en prácticamente todo momento pues tiene que ir seleccionando puntos del borde de la región a segmentar en cada corte.

Se hace uso del algoritmo de \textit{Dijkstra} para calcular el camino de coste mínimo entre el punto seleccionado por el usuario y uno anterior. Hace uso de la componente gradiente para dar coste a los nodos del grafo \cite{mortensen95}.

\begin{figure}[H]
	\centering
	\includegraphics[width=9cm]{imagenes/desarrollo/livewire}
	\caption{Ejemplo de hígado en un corte segmentado usando \textit{livewire} \cite{toennies12}}
	\label{fig:desarrollo/livewire}
\end{figure}

Este método es quizás el más preciso, pero para obtener un resultado hace falta un usuario delante del ordenador recortando todos y cada uno de los cortes del conjunto de datos volumétrico.

\section{Plataforma de desarrollo}

TODO

\subsection{Instalación y configuración}

TODO

\section{Fases de desarrollo}

Inicialmente se planteó un desarrollo en tres partes bien diferenciadas e independientes, aunque una de ellas podría ayudar a lograr mejores resultados en la siguiente. 

Hablo de la de pre-procesamiento de datos (correspondiente a la etapa de filtrado del flujo de representación de datos volumétricos) en la que se marcó el objetivo de reducir el ruido, específicamente el que producían los objetos metálicos. 

Gracias a esto sería más fácil detectar las distintas piezas de madera para la sección de subdivisión de piezas de madera (correspondiente a la etapa de segmentación). 

Por último, ya con estas dos secciones terminadas, se pasaría a añadir herramientas de documentación que ayudarían al restaurador a realizar sus tareas de documentación en la propia aplicación sin necesidad de utilizar herramientas externas.

\subsection{Pre-procesamiento de datos}

El objetivo aquí era la reducción del ruido que se podría encontrar en las imágenes. Como ya se ha explicado anteriormente, hay filtros básicos que nos permiten reducirlo. El filtro media y \textit{gaussiano} nos ayuda a suavizar y el filtro mediana a acabar con \textit{outliers} y ruido de tipo \textit{salt-and-pepper} en general. Aunque en los datos de prueba en los que trabajamos no hay ruido de tipo \textit{salt-and-pepper} es importante proveer este filtro pues en otras imágenes podría haber y es el más efectivo a la hora de acabar con él.

Para no tener que crear las matrices de convolución manualmente \textit{reinventando la rueda}, se hizo uso de una librería que nació precisamente para realizar todas las tareas previas al renderizado de volúmenes y que cuenta con una gama de filtros ya implementados. Hablo de ITK \cite{itk}. Una librería de código abierto de la misma compañía que VTK, \textit{Kitware}.

Esta librería incluye los tres filtros citados con anterioridad:

\begin{itemize}
	\item \textbf{Media}: Usando la clase \texttt{itkMeanImageFilter} pasando como parámetro el tamaño del vecindario.
	\item \textbf{Mediana}: Usando la clase \texttt{itkMedianImageFilter} pasando como parámetro el tamaño del vecindario.
	\item \textbf{\textit{Gaussiano}}: Usando la clase \texttt{itkBinomialBlurImageFilter} pasando como parámetro el número de repeticiones a realizar.
\end{itemize}

Para poder hacer el filtrado hace falta transformar la imagen actual que se encuentra en el formato utilizado en VTK para ser renderizada al formato de ITK. Y una vez realizado el filtro hacer el paso opuesto. Para ello hay que utilizar las clases \texttt{itkVTKImageToImageFilter} e \texttt{itkImageToVTKImageFilter}.

A continuación se presenta un pequeño \textit{script} con los pasos a seguir para realizar el filtrado en una imagen en VTK usando filtros de ITK (Código \ref{code:desarrollo/vtk-itk-filtro}):

\begin{lstlisting}[style=C, label=code:desarrollo/vtk-itk-filtro, caption={\textit{script} para usar el filtro media de ITK en una imagen en VTK}]
// Definir tipo de imagen ITK usando signed short y 3 dimensiones
typedef signed short PixelType;
const unsigned int Dimension = 3;
typedef itk::Image<PixelType,Dimension> ImageType;

// Crear pipeline para pasar una imagen VTK a ITK
typedef itk::VTKImageToImageFilter<ImageType> VTKImageToImageType;
VTKImageToImageType::Pointer vtkImageToImage = VTKImageToImageType::New();
vtkImageToImage->SetInput(imageData);
vtkImageToImage->Update();

// Filtrar la imagen ITK usando un filtro media
typedef itk::MeanImageFilter<ImageType,ImageType> MeanImageFilterType;
MeanImageFilterType::Pointer meanFilter = MeanImageFilterType::New();
meanFilter->SetInput(vtkImageToImage->GetOutput());
meanFilter->SetRadius(radius);
meanFilter->Update();

// Pasara de imagen ITK filtrada a VTK
typedef itk::ImageToVTKImageFilter<ImageType> ImageToVTKImageType;
ImageToVTKImageType::Pointer imageToVTKImage = ImageToVTKImageType::New();
imageToVTKImage->SetInput(meanFilter->GetOutput());
imageToVTKImage->Update();

// Actualizar la instancia de la imagen VTK
imageData->DeepCopy(imageToVTKImage->GetOutput());
imageData->Modified();
\end{lstlisting}

La integración con la interfaz es simple e intuitiva. Tan solo hay que pulsar en el botón de filtrado (Figura \ref{fig:desarrollo/gui-filtro}) y aparecerá un cuadro de diálogo donde se podrá elegir el tipo de filtro a aplicar y sus parámetros (Figura \ref{fig:desarrollo/gui-filtro-dialogo}). Una vez seleccionado se pulsa en OK y empezará a filtrar. Es un proceso largo por lo que se coloca un cuadro de diálogo informando al usuario que tenga paciencia.

\begin{figure}[H]
	\centering
	\includegraphics[width=11cm]{imagenes/desarrollo/gui-filtro}
	\caption{Botón que hay que pulsar para que aparezca el diálogo de filtrado}
	\label{fig:desarrollo/gui-filtro}
\end{figure}

\begin{figure}[H]
	\centering
	\includegraphics[width=4cm]{imagenes/desarrollo/gui-filtro-gaussiano}
	\includegraphics[width=4cm]{imagenes/desarrollo/gui-filtro-media}
	\includegraphics[width=4cm]{imagenes/desarrollo/gui-filtro-mediana}
	\caption{Cuadro de diálogo con los filtros posibles. Se muestran tres imágenes, una por cada pestaña abierta para mostrar el parámetro que hay que establecer para aplicar el filtro}
	\label{fig:desarrollo/gui-filtro-dialogo}
\end{figure}

Sin embargo ningún \textit{pipeline} de los filtros citados anteriormente ha ayudado a eliminar el ruido producido por los elementos metálicos (Figura \ref{fig:desarrollo/ruido-clavo}).

\begin{figure}[H]
	\centering
	\includegraphics[width=8cm]{imagenes/desarrollo/ruido-clavo}
	\caption{Ruido producido por los elementos metálicos}
	\label{fig:desarrollo/ruido-clavo}
\end{figure}

Y es que ningún filtro estudiado sería capaz de reducir este ruido. Pues el tamaño de vecindario que se debería aplicar sería demasiado grande y suavizaría tanto la imagen que se perderían tantos detalles que no parecería ni la imagen original.

Hay estudios que aseguran haber reducido este ruido en imágenes médicas con pequeños implantes metálicos \cite{deman98} \cite{watzke04}, pero no detallan el comportamiento de su algoritmo para reproducirlo y además, en las imágenes de prueba que utilizan el ruido es ínfimo comparado con el caso de nuestros datos.

Otro estudio más reciente \cite{boas12} usando escáneres más modernos aseguran que retocando los parámetros de este podría eliminarse prácticamente por completo. Por lo que, resolver este problema pasa a ser responsabilidad de la primera fase del flujo del modelado de conjuntos volumétrico, la obtención de datos.

\subsection{Subdivisión de piezas de madera}

TODO

\subsection{Herramientas de documentación}

TODO

\subsection{Últimas mejoras}

TODO

\section{Problemas y soluciones}

TODO
\chapter{Resultados}

En este capítulo se van a mostrar resultados obtenidos con las siguientes funcionalidades implementadas: Filtrado, segmentación y documentación.

Para realizar las pruebas se han utilizado la esculturas de Inmaculada Concepción y San Juan Evangelista (Figura \ref{fig:resultados/esculturas}), ambas patrimonio de la Universidad de Granada y cuyos datos DICOM han sido proporcionados por el proyecto de Portal Virtual de Patrimonio de las Universidades Andaluzas, coordinado por la Universidad de Granada.

\begin{figure}[H]
	\centering
	\includegraphics[width=7cm]{imagenes/resultados/esculturas}
	\caption{Esculturas utilizadas para realizar las pruebas. Inmaculada Concepción (izquierda) y San Juan Evangelista (derecha)}
	\label{fig:resultados/esculturas}
\end{figure}

\section{Filtrado}

Se implementaron los filtros de reducción de ruido \textit{gaussiano}, media y mediana dando la posibilidad al usuario a elegir ciertos parámetros.

Para probar estos filtros se ha utilizado la escultura de San Juan Evangelista que es la que más ruido presentaba.

\subsection{Filtro \textit{gaussiano}}

El filtro \textit{gaussiano} es uno de los filtros de suavizado más utilizados. A continuación se va a mostrar los resultados obtenidos con éste para 1, 2 y 3 repeticiones (Figura \ref{fig:resultados/filtrado/gaussiano}).

\begin{figure}[H]
	\centering
	\includegraphics[width=6cm]{imagenes/resultados/filtrado/original}
	\includegraphics[width=6cm]{imagenes/resultados/filtrado/gaussiano-1}
	\includegraphics[width=6cm]{imagenes/resultados/filtrado/gaussiano-2}
	\includegraphics[width=6cm]{imagenes/resultados/filtrado/gaussiano-3}
	\caption{De izquierda a derecha y arriba a abajo: figura original y aplicando el filtro \textit{gaussiano} con 1, 2 y 3 repeticiones}
	\label{fig:resultados/filtrado/gaussiano}
\end{figure}

Se puede observar como apenas hay diferencia entre la original y la que se le ha aplicado el filtro con una sola repetición.

Con la que mejor resultado se obtiene es con la que se le aplica 2 repeticiones, porque con 3 ya empieza a suavizarse de más.

\subsection{Filtro media}

El filtro media es un filtro de suavizado bastante agresivo que usa una convolución y donde el único parámetro que se puede utilizar es el tamaño del vecindario para la convolución. A continuación se va a mostrar los resultados obtenidos con éste para vecindarios de 3x3, 5x5 y 7x7 (Figura \ref{fig:resultados/filtrado/media}).

\begin{figure}[H]
	\centering
	\includegraphics[width=6cm]{imagenes/resultados/filtrado/original}
	\includegraphics[width=6cm]{imagenes/resultados/filtrado/media-3}
	\includegraphics[width=6cm]{imagenes/resultados/filtrado/media-5}
	\includegraphics[width=6cm]{imagenes/resultados/filtrado/media-7}
	\caption{De izquierda a derecha y arriba a abajo: figura original y aplicando el filtro media con vecindarios de 3x3, 5x5 y 7x7}
	\label{fig:resultados/filtrado/media}
\end{figure}

Se puede ver como este filtro es bastante más agresivo que el \textit{gaussiano} y se suaviza de más. Por tanto, habría que utilizarlo para imágenes con mucho más ruido.

\subsection{Filtro mediana}

El filtro mediana se utiliza para reducir el ruido de tipo \textit{salt-and-pepper}. Nuestras imágenes no sufren de este tipo de ruido. No obstante se ha probado el filtro con vecindarios de 3x3, 5x5 y 7x7 (Figura \ref{fig:resultados/filtrado/mediana}).

\begin{figure}[H]
	\centering
	\includegraphics[width=6cm]{imagenes/resultados/filtrado/original}
	\includegraphics[width=6cm]{imagenes/resultados/filtrado/mediana-3}
	\includegraphics[width=6cm]{imagenes/resultados/filtrado/mediana-5}
	\includegraphics[width=6cm]{imagenes/resultados/filtrado/mediana-7}
	\caption{De izquierda a derecha y arriba a abajo: figura original y aplicando el filtro mediana con vecindarios de 3x3, 5x5 y 7x7}
	\label{fig:resultados/filtrado/mediana}
\end{figure}

Los resultados con este filtro son incluso más agresivos que los obtenidos con el de media. Pero es lógico al ser un filtro creado para reducir un tipo de ruido que no presentan nuestras imágenes.

\section{Segmentación}

Se ha probado el método de segmentación propuesto para separar las piezas del embón de ambas esculturas.

\subsection{Inmaculada Concepción}

En la Inmaculada Concepción hay dos piezas principales en el embón a derecha e izquierda que se puede contemplar muy bien en cualquier corte axial desde la altura del pecho hasta abajo. Por tanto se podría usar como semilla cualquiera de estos cortes pues seguramente encuentre la línea que separa ambas partes entre las líneas rectas más grandes encontradas en este corte.

Se ha utilizado uno de estos cortes y se ha encontrado fácilmente esta línea (Figura \ref{fig:resultados/segmentacion/inmaculada-concepcion/seleccion-linea}).

\begin{figure}[H]
	\centering
	\includegraphics[width=12cm]{imagenes/resultados/segmentacion/inmaculada-concepcion/seleccion-linea}
	\caption{Líneas encontradas en un corte a la altura de la cintura donde se ve la línea (verde) que separa las dos piezas del embón}
	\label{fig:resultados/segmentacion/inmaculada-concepcion/seleccion-linea}
\end{figure}

Esta línea divide perfectamente en dos la figura obteniendo los resultados de a continuación (Figuras \ref{fig:resultados/segmentacion/inmaculada-concepcion/dcha} y \ref{fig:resultados/segmentacion/inmaculada-concepcion/dcha}).

\begin{figure}[H]
	\centering
	\includegraphics[width=6cm]{imagenes/resultados/segmentacion/inmaculada-concepcion/dcha-3d}
	\includegraphics[width=6cm]{imagenes/resultados/segmentacion/inmaculada-concepcion/dcha-corte}
	\caption{Volumen con la pieza de la derecha y un corte donde se ve dónde ha cortado}
	\label{fig:resultados/segmentacion/inmaculada-concepcion/dcha}
\end{figure}

\begin{figure}[H]
	\centering
	\includegraphics[width=6cm]{imagenes/resultados/segmentacion/inmaculada-concepcion/izda-3d}
	\includegraphics[width=6cm]{imagenes/resultados/segmentacion/inmaculada-concepcion/izda-corte}
	\caption{Volumen con la pieza de la izquierda y un corte donde se ve dónde ha cortado}
	\label{fig:resultados/segmentacion/inmaculada-concepcion/izda}
\end{figure}

\subsection{San Juan Evangelista}

En el San Juan Evangelista hay dos piezas principales en el embón una muy grande frontal y otra bastante más pequeña trasera. Esta pequeña no se puede ver por la zona de la cabeza y en la zona de la cintura hay un clavo que hace que no se diferencie muy bien la línea recta. Sin embargo, por la zona del pecho se puede encontrar una buena semilla donde se detecte la línea que corta ambas piezas de madera.

Se ha utilizado un corte por esta zona y se ha encontrado esta línea (Figura \ref{fig:resultados/segmentacion/inmaculada-concepcion/seleccion-linea}).

\begin{figure}[H]
	\centering
	\includegraphics[width=12cm]{imagenes/resultados/segmentacion/san-juan-evangelista/seleccion-linea}
	\caption{Líneas encontradas en un corte a la altura de la cintura donde se ve la línea (azul) que separa las dos piezas del embón}
	\label{fig:resultados/segmentacion/san-juan-evangelista/seleccion-linea}
\end{figure}

Esta línea divide perfectamente en dos la figura obteniendo los resultados de a continuación (Figuras \ref{fig:resultados/segmentacion/san-juan-evangelista/frontal} y \ref{fig:resultados/segmentacion/san-juan-evangelista/trasero}).

\begin{figure}[H]
	\centering
	\includegraphics[width=6cm]{imagenes/resultados/segmentacion/san-juan-evangelista/frontal-3d}
	\includegraphics[width=6cm]{imagenes/resultados/segmentacion/san-juan-evangelista/frontal-corte}
	\caption{Volumen con la pieza frontal y un corte donde se ve dónde ha cortado}
	\label{fig:resultados/segmentacion/san-juan-evangelista/frontal}
\end{figure}

\begin{figure}[H]
	\centering
	\includegraphics[width=6cm]{imagenes/resultados/segmentacion/san-juan-evangelista/trasero-3d}
	\includegraphics[width=6cm]{imagenes/resultados/segmentacion/san-juan-evangelista/trasero-corte}
	\caption{Volumen con la pieza trasera y un corte donde se ve dónde ha cortado}
	\label{fig:resultados/segmentacion/san-juan-evangelista/trasero}
\end{figure}
\chapter{Conclusiones y trabajos futuros}

Lala

\bibliography{bibliografia/bibliografia}
\addcontentsline{toc}{chapter}{Bibliografía}
\bibliographystyle{plain}

\end{document}
