\chapter*{}

\input{portada/portada_2}

\cleardoublepage
\thispagestyle{empty}

\begin{center}
{\large\bfseries \myTitle: \mySubtitle}\\
\end{center}

\begin{center}
\myName \\
\end{center}

\vspace{0.7cm}
\noindent{\textbf{Palabras clave}: informática gráfica, volúmenes, filtros, segmentación, tomografía computarizada, escultura policromada de madera, conservación y restauración, restaurador de arte}\\

\vspace{0.7cm}
\noindent{\textbf{Resumen}}\\

Tras resolver el problema del renderizado de conjuntos de datos volumétricos y crear un software sencillo para estas tareas, se pretende continuar el desarrollo de éste enfocándose en otros tres objetivos claros. Dos de ellos son completar las otras dos fases de modelado de volúmenes previas al renderizado: el filtrado y la segmentación. Y el otro es proveer al software de herramientas para poder realizar tareas de documentación por parte de los usuarios finales de éste. En cuanto al filtrado, se van a proporcionar filtros básicos de reducción de ruido para no renderizas los datos en crudo que nos devuelve el escáner. El problema de segmentación que se tratará de resolver será el de dividir la figura en los distintos trozos de madera que la conforman. Por último se creará un entorno de documentación integrado en el software para poder realizar cualquier apunte dentro de la aplicación. Se integrarán widgets para realizar medidas de distancias y ángulos y para hacer anotaciones de texto en distintos cortes generados. Todas estas tareas se dividirán en distintas tareas y se seguirá un desarrollo ágil usando las herramientas que me proporciona GitHub para la gestión de un proyecto. Una vez terminado el desarrollo se realizarán una serie de pruebas y se documentarán los resultados obtenidos.

\thispagestyle{empty}

\cleardoublepage

\begin{center}
	{\large\bfseries \myTitle: \myEnglishSubtitle}\\
\end{center}

\begin{center}
	\myName \\
\end{center}

\vspace{0.7cm}
\noindent{\textbf{Keywords}: Computer Graphics, Volume, Filtering, Segmentation, Computed Tomography, Polychromed Wood Sculpture, Conservation-Restoration of Cultural Heritage, Art Curator}\\

\vspace{0.7cm}
\noindent{\textbf{Abstract}}\\

After solving the problem of volume rendering and build an application to do these tasks we will continue the development focusing on three well-differentiated objectives. Two of them are to complete the other two steps of volume modelling before the rendering: filtering and segmentation. And the other one is to provide the software of tools to make documentation tasks by the final user. To solve the filtering objective we will provide basic reduction noise filters to apply the volumetric dataset before rendering it. Regarding the segmentation, we will try to solve the problem of divide the original figure into the different pieces of wood it is composed by. Finally we will build an integrated documentation environment in the software. We will integrate widgets to make measurements of distances and angles and to make text annotations in different generated slices. All these tasks will be divided into sub-tasks and we will follow an agile development by using the tools provided by GitHub to manage a project. Once we finish the development, we will make some tests and we will documentate the obtained results.

\chapter*{}
\thispagestyle{empty}

\noindent\rule[-1ex]{\textwidth}{2pt}\\[4.5ex]

Yo, \textbf{\myName}, alumno de la titulación \myDegree de la \textbf{\myFaculty}, con DNI 75926571Y, autorizo la ubicación de la siguiente copia de mi Trabajo Fin de Máster en la biblioteca del centro para que pueda ser consultada por las personas que lo deseen.

\vspace{6cm}

\noindent Fdo: \myName

\vspace{2cm}

\begin{flushright}
\myLocation a \myTime.
\end{flushright}


\chapter*{}
\thispagestyle{empty}

\noindent\rule[-1ex]{\textwidth}{2pt}\\[4.5ex]

D. \textbf{\myProf}, Profesor del Área de Lenguajes y Sistemas Informáticos del \myDepartment de la \myUni.

\vspace{0.5cm}

\textbf{Informa:}

\vspace{0.5cm}

Que el presente trabajo, titulado \textit{\textbf{\myTitle, \mySubtitle}}, ha sido realizado bajo su supervisión por \textbf{\myName}, y autorizamos la defensa de dicho trabajo ante el tribunal que corresponda.

\vspace{0.5cm}

Y para que conste, expiden y firman el presente informe en \myLocation a \myTime.

\vspace{1cm}

\textbf{El director:}

\vspace{5cm}

\noindent \textbf{\myProf}

\chapter*{Agradecimientos}
\thispagestyle{empty}

\vspace{1cm}

En primer lugar, agradecer a todos aquellos que han hecho esto posible: 
\\

A mi tutor, Javier Melero, que confió en mi para este proyecto cuando hace ya casi tres años fui a su despacho en busca de un trabajo de fin de grado que ha podido ser continuado como trabajo fin de máster. 
\\

Al Portal Virtual de Patrimonio de las Universidades Andaluzas por la cesión de los modelos de las esculturas de Inmaculada Concepción y San Juan Evangelista que han sido utilizadas para probar el software. 
\\

Y a Concha y Amparo de Artemisia Gestión de Patrimonio por ofrecerme parte de su tiempo y proporcionar su conocimiento del tema.
\\

Por otra parte, también querría que tuviesen un apartado en esta sección ese grupo fantástico de compañeros que he tenido durante el máster. 
\\

Y cómo no, a todos los que me han animado a seguir cuando apenas quedaban fuerzas.