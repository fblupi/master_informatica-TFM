\chapter{Análisis}

En este capítulo se describirán las distintas historias de usuario que han sido implementadas en el software.

\section{Historias de usuario}

\subsection{\textit{Product backlog}}

A continuación se muestra el listado de historias de usuario (\textit{Product Backlog}) completo separadas por módulo, y para cada historia de usuario sus dependencias, estimación (en puntos de historia) y prioridad.

\begin{longtable} {r l c c c}
	\hline
	\#	&	\textbf{Descripción}					&	\textbf{Dep.}	&	\textbf{Est.}	&	\textbf{Prio.}	\\
	\hline \hline
	\endhead
	\multicolumn{5}{l}{\textbf{Auxiliar}} \\
	\hline 
	1.1.	&	Exportar volumen					&	-				&	12				&	1	\\
	\hline
	1.2.	&	Importar volumen					&	1.1.			&	8				&	5	\\
	\hline
	\multicolumn{5}{l}{\textbf{Pre-procesamiento}} \\
	\hline 
	2.1.	&	Filtro \textit{gaussiano}			&	-				&	24				&	1	\\
	\hline
	2.2.	&	Filtro media						&	-				&	10				&	3	\\
	\hline
	2.3.	&	Filtro mediana						&	-				&	8				&	3	\\
	\hline
	\multicolumn{5}{l}{\textbf{Segmentación}} \\
	\hline 
	3.1.	&	Segmentar pieza de madera			&	1.1.			&	64				&	1	\\
	\hline
	\multicolumn{5}{l}{\textbf{Documentación}} \\
	\hline 
	4.1.	&	Crear ROD							&	-				&	4				&	1	\\
	\hline
	4.2.	&	Eliminar ROD						&	4.1.			&	4				&	3	\\
	\hline
	4.3.	&	Exportar ROD						&	4.1.			&	2				&	2	\\
	\hline
	4.4.	&	Importar ROD						&	4.3.			&	3				&	2	\\
	\hline
	4.5.	&	Cambiar ROD							&	4.1.			&	3				&	1	\\
	\hline
	4.6.	&	Crear regla							&	-				&	3				&	2	\\
	\hline
	4.7.	&	Eliminar regla						&	4.6				&	3				&	4	\\
	\hline
	4.8.	&	Editar regla						&	4.6				&	1				&	5	\\
	\hline
	4.9.	&	Ocultar regla						&	4.6				&	2				&	6	\\
	\hline
	4.10.	&	Mostrar regla						&	4.9				&	1				&	6	\\
	\hline
	4.11.	&	Crear transportador de ángulos		&	-				&	3				&	3	\\
	\hline
	4.12.	&	Eliminar transportador de ángulos	&	4.11.			&	3				&	5	\\
	\hline
	4.13.	&	Editar transportador de ángulos		&	4.11.			&	1				&	6	\\
	\hline
	4.14.	&	Ocultar transportador de ángulos	&	4.11.			&	2				&	7	\\
	\hline
	4.15.	&	Mostrar transportador de ángulos	&	4.14.			&	1				&	7	\\
	\hline
	4.16.	&	Crear nota							&	-				&	4				&	2	\\
	\hline
	4.17.	&	Eliminar nota						&	4.16.			&	2				&	4	\\
	\hline
	4.18.	&	Editar nota							&	4.16.			&	3				&	5	\\
	\hline
	4.19.	&	Ocultar nota						&	4.16.			&	2				&	6	\\
	\hline
	4.20.	&	Mostrar nota						&	4.19.			&	1				&	6	\\
	\hline
	\multicolumn{5}{l}{\textbf{Mejoras en código}} \\
	\hline
	5.1.	&	Internacionalización				&	-				&	6				&	8	\\
	\hline
	\\
	\caption{Historias de usuario}
	\label{tab:analisis/hus}
\end{longtable}

Se han estimado un total de 180 PH (Puntos de historia) y se habían planificado unas 360 horas de trabajo. Por lo que cada PH corresponde aproximadamente a 2 horas.

Al haber registrado 312 horas finalmente, he determinado que mi velocidad ha sido aproximadamente 1,73 horas/PH, un poco más rápido que lo planificado.

\subsection{Tarjetas de las historias de usuario}

A continuación se incluye una descripción completa de las historias de usuario incluyendo una descripción de ésta y sus correspondientes criterios de aceptación.

\subsubsection{Auxiliar}

\begin{table}[H]
	\begin{center}
		\begin{tabular} {l|c|l}
			\hline
			1.1. & \multicolumn{2}{c}{Exportar volumen} \\ \noalign{\hrule height 1pt}
			\multicolumn{3}{l}{Descripción} \\ \hline
			\multicolumn{3}{p{12cm}}{Se debe proveer a la aplicación de la funcionalidad necesaria para exportar el volumen que se ha cargado y editado para poder utilizarlo de nuevo posteriormente sin tener que volver a editarlo. Para ello se tratará de utilizar el formato propio de VTK para almacenar volúmenes, VTI, que hace uso de XML. Se le mostrará al usuario un diálogo donde elegirá la carpeta y el nombre del archivo.} \\ \noalign{\hrule height 1pt}
			\multicolumn{2}{l|}{Estimación} & 12 \\ \hline
			\multicolumn{2}{l|}{Prioridad} & 1 \\ \hline
			\multicolumn{2}{l|}{Dependencias} & - \\ \noalign{\hrule height 1pt}
			\multicolumn{3}{l}{Pruebas de aceptación} \\ \hline
			\multicolumn{3}{p{12cm}}{ - El usuario no elige nombre y se guardará el archivo con un nombre por defecto ne la carpeta elegida.} \\ 
			\multicolumn{3}{p{12cm}}{ - El usuario elige el nombre y se guardará el archivo con el nombre elegido en la carpeta elegida.} \\ 
			\hline
		\end{tabular}
	\end{center}
	\caption{Historia de usuario - Exportar volumen}
	\label{tab:analisis/hu-exportar-volumen}
\end{table}

\begin{table}[H]
	\begin{center}
		\begin{tabular} {l|c|l}
			\hline
			1.2. & \multicolumn{2}{c}{Importar volumen} \\ \noalign{\hrule height 1pt}
			\multicolumn{3}{l}{Descripción} \\ \hline
			\multicolumn{3}{p{12cm}}{El \textit{software} debe poder leer ficheros en formato VTI como los previamente exportados. Para ello el usuario deberá poder elegir un archivo en un diálogo donde se filtrarán los ficheros para que se muestren solo los que tienen la extensión VTI.} \\ \noalign{\hrule height 1pt}
			\multicolumn{2}{l|}{Estimación} & 8 \\ \hline
			\multicolumn{2}{l|}{Prioridad} & 5 \\ \hline
			\multicolumn{2}{l|}{Dependencias} & 1.1. \\ \noalign{\hrule height 1pt}
			\multicolumn{3}{l}{Pruebas de aceptación} \\ \hline
			\multicolumn{3}{p{12cm}}{ - Si el usuario lee un archivo que no es VTI no lo podrá visualizar.} \\
			\multicolumn{3}{p{12cm}}{ - Si el usuario elige un archivo VTI correcto se importará y podrá empezar a utilizarlo.} \\ \hline
		\end{tabular}
	\end{center}
	\caption{Historia de usuario - Importar volumen}
	\label{tab:analisis/hu-importar-volumen}
\end{table}

\subsubsection{Pre-procesamiento}

\begin{table}[H]
	\begin{center}
		\begin{tabular} {l|c|l}
			\hline
			2.1. & \multicolumn{2}{c}{Filtro \textit{gaussiano}} \\ \noalign{\hrule height 1pt}
			\multicolumn{3}{l}{Descripción} \\ \hline
			\multicolumn{3}{p{12cm}}{Se debe poder aplicar un filtro \textit{gaussiano} al volumen para reducir el ruido. Para ello el usuario elegirá el número de repeticiones que se realizarán (de 1 a 5). Se utilizará la librería ITK para aplicar este filtro.} \\ \noalign{\hrule height 1pt}
			\multicolumn{2}{l|}{Estimación} & 24 \\ \hline
			\multicolumn{2}{l|}{Prioridad} & 1 \\ \hline
			\multicolumn{2}{l|}{Dependencias} & - \\ \noalign{\hrule height 1pt}
			\multicolumn{3}{l}{Pruebas de aceptación} \\ \hline
			\multicolumn{3}{p{12cm}}{ - El usuario selecciona el filtrado y los parámetros deseados y se realiza el filtrado en el volumen, el resultado será un volumen más suavizado.} \\ \hline
		\end{tabular}
	\end{center}
	\caption{Historia de usuario - Filtro \textit{gaussiano}}
	\label{tab:analisis/hu-filtro-gaussiano}
\end{table}

\begin{table}[H]
	\begin{center}
		\begin{tabular} {l|c|l}
			\hline
			2.2. & \multicolumn{2}{c}{Filtro media} \\ \noalign{\hrule height 1pt}
			\multicolumn{3}{l}{Descripción} \\ \hline
			\multicolumn{3}{p{12cm}}{Se debe poder aplicar un filtro media al volumen para reducir ruido. Para ello el usuario elegirá el tamaño del vecindario (3x3, 5x5 o 7x7). Se utilizará la librería ITK para aplicar este filtro.} \\ \noalign{\hrule height 1pt}
			\multicolumn{2}{l|}{Estimación} & 10 \\ \hline
			\multicolumn{2}{l|}{Prioridad} & 3 \\ \hline
			\multicolumn{2}{l|}{Dependencias} & - \\ \noalign{\hrule height 1pt}
			\multicolumn{3}{l}{Pruebas de aceptación} \\ \hline
			\multicolumn{3}{p{12cm}}{ - El usuario selecciona el filtrado y los parámetros deseados y se realiza el filtrado en el volumen, el resultado será un volumen más suavizado.} \\ \hline
		\end{tabular}
	\end{center}
	\caption{Historia de usuario - Filtro media}
	\label{tab:analisis/hu-filtro-media}
\end{table}

\begin{table}[H]
	\begin{center}
		\begin{tabular} {l|c|l}
			\hline
			2.3. & \multicolumn{2}{c}{Filtro mediana} \\ \noalign{\hrule height 1pt}
			\multicolumn{3}{l}{Descripción} \\ \hline
			\multicolumn{3}{p{12cm}}{Se debe poder aplicar un filtro mediana al volumen para reducir ruido. Para ello el usuario elegirá el tamaño del vecindario (3x3, 5x5 o 7x7). Se utilizará la librería ITK para aplicar este filtro.} \\ \noalign{\hrule height 1pt}
			\multicolumn{2}{l|}{Estimación} & 8 \\ \hline
			\multicolumn{2}{l|}{Prioridad} & 3 \\ \hline
			\multicolumn{2}{l|}{Dependencias} & - \\ \noalign{\hrule height 1pt}
			\multicolumn{3}{l}{Pruebas de aceptación} \\ \hline
			\multicolumn{3}{p{12cm}}{ - El usuario selecciona el filtrado y los parámetros deseados y se realiza el filtrado en el volumen. Esto reducirá el ruido de tipo \textit{salt-and-pepper}.} \\ \hline
		\end{tabular}
	\end{center}
	\caption{Historia de usuario - Filtro mediana}
	\label{tab:analisis/hu-filtro-mediana}
\end{table}

\subsubsection{Segmentación}

\begin{table}[H]
	\begin{center}
		\begin{tabular} {l|c|l}
			\hline
			3.1. & \multicolumn{2}{c}{Segmentar pieza de madera} \\ \noalign{\hrule height 1pt}
			\multicolumn{3}{l}{Descripción} \\ \hline
			\multicolumn{3}{p{12cm}}{.} \\ \noalign{\hrule height 1pt}
			\multicolumn{2}{l|}{Estimación} & 64 \\ \hline
			\multicolumn{2}{l|}{Prioridad} & 1 \\ \hline
			\multicolumn{2}{l|}{Dependencias} & 1.1. \\ \noalign{\hrule height 1pt}
			\multicolumn{3}{l}{Pruebas de aceptación} \\ \hline
			\multicolumn{3}{p{12cm}}{ - .} \\ \hline
		\end{tabular}
	\end{center}
	\caption{Historia de usuario - Segmentar pieza de madera}
	\label{tab:analisis/hu-segmentar-pieza-de-madera}
\end{table}

\subsubsection{Documentación}

\begin{table}[H]
	\begin{center}
		\begin{tabular} {l|c|l}
			\hline
			4.1. & \multicolumn{2}{c}{Crear ROD} \\ \noalign{\hrule height 1pt}
			\multicolumn{3}{l}{Descripción} \\ \hline
			\multicolumn{3}{p{12cm}}{.} \\ \noalign{\hrule height 1pt}
			\multicolumn{2}{l|}{Estimación} & 4 \\ \hline
			\multicolumn{2}{l|}{Prioridad} & 1 \\ \hline
			\multicolumn{2}{l|}{Dependencias} & - \\ \noalign{\hrule height 1pt}
			\multicolumn{3}{l}{Pruebas de aceptación} \\ \hline
			\multicolumn{3}{p{12cm}}{ - .} \\ \hline
		\end{tabular}
	\end{center}
	\caption{Historia de usuario - Crear ROD}
	\label{tab:analisis/hu-crear-rod}
\end{table}

\begin{table}[H]
	\begin{center}
		\begin{tabular} {l|c|l}
			\hline
			4.2. & \multicolumn{2}{c}{Eliminar ROD} \\ \noalign{\hrule height 1pt}
			\multicolumn{3}{l}{Descripción} \\ \hline
			\multicolumn{3}{p{12cm}}{.} \\ \noalign{\hrule height 1pt}
			\multicolumn{2}{l|}{Estimación} & 4 \\ \hline
			\multicolumn{2}{l|}{Prioridad} & 3 \\ \hline
			\multicolumn{2}{l|}{Dependencias} & 4.1. \\ \noalign{\hrule height 1pt}
			\multicolumn{3}{l}{Pruebas de aceptación} \\ \hline
			\multicolumn{3}{p{12cm}}{ - .} \\ \hline
		\end{tabular}
	\end{center}
	\caption{Historia de usuario - Eliminar ROD}
	\label{tab:analisis/hu-eliminar-rod}
\end{table}

\begin{table}[H]
	\begin{center}
		\begin{tabular} {l|c|l}
			\hline
			4.3. & \multicolumn{2}{c}{Exportar ROD} \\ \noalign{\hrule height 1pt}
			\multicolumn{3}{l}{Descripción} \\ \hline
			\multicolumn{3}{p{12cm}}{.} \\ \noalign{\hrule height 1pt}
			\multicolumn{2}{l|}{Estimación} & 2 \\ \hline
			\multicolumn{2}{l|}{Prioridad} & 2 \\ \hline
			\multicolumn{2}{l|}{Dependencias} & 4.1. \\ \noalign{\hrule height 1pt}
			\multicolumn{3}{l}{Pruebas de aceptación} \\ \hline
			\multicolumn{3}{p{12cm}}{ - .} \\ \hline
		\end{tabular}
	\end{center}
	\caption{Historia de usuario - Exportar ROD}
	\label{tab:analisis/hu-exportar-rod}
\end{table}

\begin{table}[H]
	\begin{center}
		\begin{tabular} {l|c|l}
			\hline
			4.4. & \multicolumn{2}{c}{Importar ROD} \\ \noalign{\hrule height 1pt}
			\multicolumn{3}{l}{Descripción} \\ \hline
			\multicolumn{3}{p{12cm}}{.} \\ \noalign{\hrule height 1pt}
			\multicolumn{2}{l|}{Estimación} & 3 \\ \hline
			\multicolumn{2}{l|}{Prioridad} & 2 \\ \hline
			\multicolumn{2}{l|}{Dependencias} & 4.3. \\ \noalign{\hrule height 1pt}
			\multicolumn{3}{l}{Pruebas de aceptación} \\ \hline
			\multicolumn{3}{p{12cm}}{ - .} \\ \hline
		\end{tabular}
	\end{center}
	\caption{Historia de usuario - Importar ROD}
	\label{tab:analisis/hu-importar-rod}
\end{table}

\begin{table}[H]
	\begin{center}
		\begin{tabular} {l|c|l}
			\hline
			4.5. & \multicolumn{2}{c}{Cambiar ROD} \\ \noalign{\hrule height 1pt}
			\multicolumn{3}{l}{Descripción} \\ \hline
			\multicolumn{3}{p{12cm}}{.} \\ \noalign{\hrule height 1pt}
			\multicolumn{2}{l|}{Estimación} & 3 \\ \hline
			\multicolumn{2}{l|}{Prioridad} & 1 \\ \hline
			\multicolumn{2}{l|}{Dependencias} & 4.1. \\ \noalign{\hrule height 1pt}
			\multicolumn{3}{l}{Pruebas de aceptación} \\ \hline
			\multicolumn{3}{p{12cm}}{ - .} \\ \hline
		\end{tabular}
	\end{center}
	\caption{Historia de usuario - Cambiar ROD}
\label{tab:analisis/hu-cambiar-rod}
\end{table}

\begin{table}[H]
	\begin{center}
		\begin{tabular} {l|c|l}
			\hline
			4.6. & \multicolumn{2}{c}{Crear regla} \\ \noalign{\hrule height 1pt}
			\multicolumn{3}{l}{Descripción} \\ \hline
			\multicolumn{3}{p{12cm}}{.} \\ \noalign{\hrule height 1pt}
			\multicolumn{2}{l|}{Estimación} & 3 \\ \hline
			\multicolumn{2}{l|}{Prioridad} & 2 \\ \hline
			\multicolumn{2}{l|}{Dependencias} & - \\ \noalign{\hrule height 1pt}
			\multicolumn{3}{l}{Pruebas de aceptación} \\ \hline
			\multicolumn{3}{p{12cm}}{ - .} \\ \hline
		\end{tabular}
	\end{center}
	\caption{Historia de usuario - Crear regla}
	\label{tab:analisis/hu-crear-regla}
\end{table}

\begin{table}[H]
	\begin{center}
		\begin{tabular} {l|c|l}
			\hline
			4.7. & \multicolumn{2}{c}{Eliminar regla} \\ \noalign{\hrule height 1pt}
			\multicolumn{3}{l}{Descripción} \\ \hline
			\multicolumn{3}{p{12cm}}{.} \\ \noalign{\hrule height 1pt}
			\multicolumn{2}{l|}{Estimación} & 3 \\ \hline
			\multicolumn{2}{l|}{Prioridad} & 4 \\ \hline
			\multicolumn{2}{l|}{Dependencias} & 4.6. \\ \noalign{\hrule height 1pt}
			\multicolumn{3}{l}{Pruebas de aceptación} \\ \hline
			\multicolumn{3}{p{12cm}}{ - .} \\ \hline
		\end{tabular}
	\end{center}
	\caption{Historia de usuario - Eliminar regla}
	\label{tab:analisis/hu-eliminar-regla}
\end{table}

\begin{table}[H]
	\begin{center}
		\begin{tabular} {l|c|l}
			\hline
			4.8. & \multicolumn{2}{c}{Editar regla} \\ \noalign{\hrule height 1pt}
			\multicolumn{3}{l}{Descripción} \\ \hline
			\multicolumn{3}{p{12cm}}{.} \\ \noalign{\hrule height 1pt}
			\multicolumn{2}{l|}{Estimación} & 1 \\ \hline
			\multicolumn{2}{l|}{Prioridad} & 5 \\ \hline
			\multicolumn{2}{l|}{Dependencias} & 4.6. \\ \noalign{\hrule height 1pt}
			\multicolumn{3}{l}{Pruebas de aceptación} \\ \hline
			\multicolumn{3}{p{12cm}}{ - .} \\ \hline
		\end{tabular}
	\end{center}
	\caption{Historia de usuario - Editar regla}
	\label{tab:analisis/hu-editar-regla}
\end{table}

\begin{table}[H]
	\begin{center}
		\begin{tabular} {l|c|l}
			\hline
			4.9. & \multicolumn{2}{c}{Ocultar regla} \\ \noalign{\hrule height 1pt}
			\multicolumn{3}{l}{Descripción} \\ \hline
			\multicolumn{3}{p{12cm}}{.} \\ \noalign{\hrule height 1pt}
			\multicolumn{2}{l|}{Estimación} & 2 \\ \hline
			\multicolumn{2}{l|}{Prioridad} & 6 \\ \hline
			\multicolumn{2}{l|}{Dependencias} & 4.6. \\ \noalign{\hrule height 1pt}
			\multicolumn{3}{l}{Pruebas de aceptación} \\ \hline
			\multicolumn{3}{p{12cm}}{ - .} \\ \hline
		\end{tabular}
	\end{center}
	\caption{Historia de usuario - Ocultar regla}
	\label{tab:analisis/hu-ocultar-regla}
\end{table}

\begin{table}[H]
	\begin{center}
		\begin{tabular} {l|c|l}
			\hline
			4.10. & \multicolumn{2}{c}{Mostrar regla} \\ \noalign{\hrule height 1pt}
			\multicolumn{3}{l}{Descripción} \\ \hline
			\multicolumn{3}{p{12cm}}{.} \\ \noalign{\hrule height 1pt}
			\multicolumn{2}{l|}{Estimación} & 1 \\ \hline
			\multicolumn{2}{l|}{Prioridad} & 6 \\ \hline
			\multicolumn{2}{l|}{Dependencias} & 4.9. \\ \noalign{\hrule height 1pt}
			\multicolumn{3}{l}{Pruebas de aceptación} \\ \hline
			\multicolumn{3}{p{12cm}}{ - .} \\ \hline
		\end{tabular}
	\end{center}
	\caption{Historia de usuario - Mostrar regla}
	\label{tab:analisis/hu-mostrar-regla}
\end{table}

\begin{table}[H]
	\begin{center}
		\begin{tabular} {l|c|l}
			\hline
			4.11. & \multicolumn{2}{c}{Crear transportador de ángulos} \\ \noalign{\hrule height 1pt}
			\multicolumn{3}{l}{Descripción} \\ \hline
			\multicolumn{3}{p{12cm}}{.} \\ \noalign{\hrule height 1pt}
			\multicolumn{2}{l|}{Estimación} & 3 \\ \hline
			\multicolumn{2}{l|}{Prioridad} & 3 \\ \hline
			\multicolumn{2}{l|}{Dependencias} & - \\ \noalign{\hrule height 1pt}
			\multicolumn{3}{l}{Pruebas de aceptación} \\ \hline
			\multicolumn{3}{p{12cm}}{ - .} \\ \hline
		\end{tabular}
	\end{center}
	\caption{Historia de usuario - Crear transportador de ángulos}
	\label{tab:analisis/hu-crear-transportador-angulos}
\end{table}

\begin{table}[H]
	\begin{center}
		\begin{tabular} {l|c|l}
			\hline
			4.12. & \multicolumn{2}{c}{Eliminar transportador de ángulos} \\ \noalign{\hrule height 1pt}
			\multicolumn{3}{l}{Descripción} \\ \hline
			\multicolumn{3}{p{12cm}}{.} \\ \noalign{\hrule height 1pt}
			\multicolumn{2}{l|}{Estimación} & 3 \\ \hline
			\multicolumn{2}{l|}{Prioridad} & 4 \\ \hline
			\multicolumn{2}{l|}{Dependencias} & 4.11. \\ \noalign{\hrule height 1pt}
			\multicolumn{3}{l}{Pruebas de aceptación} \\ \hline
			\multicolumn{3}{p{12cm}}{ - .} \\ \hline
		\end{tabular}
	\end{center}
	\caption{Historia de usuario - Eliminar transportador de ángulos}
	\label{tab:analisis/hu-eliminar-transportador-angulos}
\end{table}

\begin{table}[H]
	\begin{center}
		\begin{tabular} {l|c|l}
			\hline
			4.13. & \multicolumn{2}{c}{Editar transportador de ángulos} \\ \noalign{\hrule height 1pt}
			\multicolumn{3}{l}{Descripción} \\ \hline
			\multicolumn{3}{p{12cm}}{.} \\ \noalign{\hrule height 1pt}
			\multicolumn{2}{l|}{Estimación} & 1 \\ \hline
			\multicolumn{2}{l|}{Prioridad} & 5 \\ \hline
			\multicolumn{2}{l|}{Dependencias} & 4.11. \\ \noalign{\hrule height 1pt}
			\multicolumn{3}{l}{Pruebas de aceptación} \\ \hline
			\multicolumn{3}{p{12cm}}{ - .} \\ \hline
		\end{tabular}
	\end{center}
	\caption{Historia de usuario - Editar transportador de ángulos}
	\label{tab:analisis/hu-editar-transportador-angulos}
\end{table}

\begin{table}[H]
	\begin{center}
		\begin{tabular} {l|c|l}
			\hline
			4.14. & \multicolumn{2}{c}{Ocultar transportador de ángulos} \\ \noalign{\hrule height 1pt}
			\multicolumn{3}{l}{Descripción} \\ \hline
			\multicolumn{3}{p{12cm}}{.} \\ \noalign{\hrule height 1pt}
			\multicolumn{2}{l|}{Estimación} & 2 \\ \hline
			\multicolumn{2}{l|}{Prioridad} & 6 \\ \hline
			\multicolumn{2}{l|}{Dependencias} & 4.11. \\ \noalign{\hrule height 1pt}
			\multicolumn{3}{l}{Pruebas de aceptación} \\ \hline
			\multicolumn{3}{p{12cm}}{ - .} \\ \hline
		\end{tabular}
	\end{center}
	\caption{Historia de usuario - Ocultar transportador de ángulos}
	\label{tab:analisis/hu-ocultar-transportador-angulos}
\end{table}

\begin{table}[H]
	\begin{center}
		\begin{tabular} {l|c|l}
			\hline
			4.15. & \multicolumn{2}{c}{Mostrar transportador de ángulos} \\ \noalign{\hrule height 1pt}
			\multicolumn{3}{l}{Descripción} \\ \hline
			\multicolumn{3}{p{12cm}}{.} \\ \noalign{\hrule height 1pt}
			\multicolumn{2}{l|}{Estimación} & 1 \\ \hline
			\multicolumn{2}{l|}{Prioridad} & 7 \\ \hline
			\multicolumn{2}{l|}{Dependencias} & 4.14. \\ \noalign{\hrule height 1pt}
			\multicolumn{3}{l}{Pruebas de aceptación} \\ \hline
			\multicolumn{3}{p{12cm}}{ - .} \\ \hline
		\end{tabular}
	\end{center}
	\caption{Historia de usuario - Mostrar transportador de ángulos}
	\label{tab:analisis/hu-mostrar-transportador-angulos}
\end{table}

\begin{table}[H]
	\begin{center}
		\begin{tabular} {l|c|l}
			\hline
			4.16. & \multicolumn{2}{c}{Crear nota} \\ \noalign{\hrule height 1pt}
			\multicolumn{3}{l}{Descripción} \\ \hline
			\multicolumn{3}{p{12cm}}{.} \\ \noalign{\hrule height 1pt}
			\multicolumn{2}{l|}{Estimación} & 4 \\ \hline
			\multicolumn{2}{l|}{Prioridad} & 2 \\ \hline
			\multicolumn{2}{l|}{Dependencias} & - \\ \noalign{\hrule height 1pt}
			\multicolumn{3}{l}{Pruebas de aceptación} \\ \hline
			\multicolumn{3}{p{12cm}}{ - .} \\ \hline
		\end{tabular}
	\end{center}
	\caption{Historia de usuario - Crear nota}
	\label{tab:analisis/hu-crear-nota}
\end{table}

\begin{table}[H]
	\begin{center}
		\begin{tabular} {l|c|l}
			\hline
			4.17. & \multicolumn{2}{c}{Eliminar nota} \\ \noalign{\hrule height 1pt}
			\multicolumn{3}{l}{Descripción} \\ \hline
			\multicolumn{3}{p{12cm}}{.} \\ \noalign{\hrule height 1pt}
			\multicolumn{2}{l|}{Estimación} & 2 \\ \hline
			\multicolumn{2}{l|}{Prioridad} & 4 \\ \hline
			\multicolumn{2}{l|}{Dependencias} & 4.16. \\ \noalign{\hrule height 1pt}
			\multicolumn{3}{l}{Pruebas de aceptación} \\ \hline
			\multicolumn{3}{p{12cm}}{ - .} \\ \hline
		\end{tabular}
	\end{center}
	\caption{Historia de usuario - Eliminar nota}
	\label{tab:analisis/hu-eliminar-nota}
\end{table}

\begin{table}[H]
	\begin{center}
		\begin{tabular} {l|c|l}
			\hline
			4.18. & \multicolumn{2}{c}{Editar nota} \\ \noalign{\hrule height 1pt}
			\multicolumn{3}{l}{Descripción} \\ \hline
			\multicolumn{3}{p{12cm}}{.} \\ \noalign{\hrule height 1pt}
			\multicolumn{2}{l|}{Estimación} & 3 \\ \hline
			\multicolumn{2}{l|}{Prioridad} & 5 \\ \hline
			\multicolumn{2}{l|}{Dependencias} & 4.16. \\ \noalign{\hrule height 1pt}
			\multicolumn{3}{l}{Pruebas de aceptación} \\ \hline
			\multicolumn{3}{p{12cm}}{ - .} \\ \hline
		\end{tabular}
	\end{center}
	\caption{Historia de usuario - Editar nota}
	\label{tab:analisis/hu-editar-nota}
\end{table}

\begin{table}[H]
	\begin{center}
		\begin{tabular} {l|c|l}
			\hline
			4.19. & \multicolumn{2}{c}{Ocultar nota} \\ \noalign{\hrule height 1pt}
			\multicolumn{3}{l}{Descripción} \\ \hline
			\multicolumn{3}{p{12cm}}{.} \\ \noalign{\hrule height 1pt}
			\multicolumn{2}{l|}{Estimación} & 2 \\ \hline
			\multicolumn{2}{l|}{Prioridad} & 6 \\ \hline
			\multicolumn{2}{l|}{Dependencias} & 4.16. \\ \noalign{\hrule height 1pt}
			\multicolumn{3}{l}{Pruebas de aceptación} \\ \hline
			\multicolumn{3}{p{12cm}}{ - .} \\ \hline
		\end{tabular}
	\end{center}
	\caption{Historia de usuario - Ocultar nota}
	\label{tab:analisis/hu-ocultar-nota}
\end{table}

\begin{table}[H]
	\begin{center}
		\begin{tabular} {l|c|l}
			\hline
			4.20. & \multicolumn{2}{c}{Mostrar nota} \\ \noalign{\hrule height 1pt}
			\multicolumn{3}{l}{Descripción} \\ \hline
			\multicolumn{3}{p{12cm}}{.} \\ \noalign{\hrule height 1pt}
			\multicolumn{2}{l|}{Estimación} & 1 \\ \hline
			\multicolumn{2}{l|}{Prioridad} & 6 \\ \hline
			\multicolumn{2}{l|}{Dependencias} & 4.19. \\ \noalign{\hrule height 1pt}
			\multicolumn{3}{l}{Pruebas de aceptación} \\ \hline
			\multicolumn{3}{p{12cm}}{ - .} \\ \hline
		\end{tabular}
	\end{center}
	\caption{Historia de usuario - Mostrar nota}
	\label{tab:analisis/hu-mostrar-nota}
\end{table}

\subsubsection{Mejoras en código}

\begin{table}[H]
	\begin{center}
		\begin{tabular} {l|c|l}
			\hline
			5.1. & \multicolumn{2}{c}{Internacionalización} \\ \noalign{\hrule height 1pt}
			\multicolumn{3}{l}{Descripción} \\ \hline
			\multicolumn{3}{p{12cm}}{El usuario tiene que tener la posibilidad de utilizar la aplicación en dos idiomas distintos, español e inglés. Para ello se utilizará el mecanismo de internacionalización que provee Qt y se generarán dos instaladores: uno en español y otro en inglés.} \\ \noalign{\hrule height 1pt}
			\multicolumn{2}{l|}{Estimación} & 6 \\ \hline
			\multicolumn{2}{l|}{Prioridad} & 8 \\ \hline
			\multicolumn{2}{l|}{Dependencias} & - \\ \noalign{\hrule height 1pt}
			\multicolumn{3}{l}{Pruebas de aceptación} \\ \hline
			\multicolumn{3}{p{12cm}}{ - El usuario ejecutará el programa en cualquiera de los idiomas y se mostrarán los textos con el idioma correspondiente.} \\ \hline
		\end{tabular}
	\end{center}
	\caption{Historia de usuario - Internacionalización}
	\label{tab:analisis/hu-internacionalizacion}
\end{table}