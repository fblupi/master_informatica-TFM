\chapter{Análisis}

En este capítulo se describirán las distintas historias de usuario que han sido implementadas en el software.

\section{Historias de usuario}

\subsection{\textit{Product backlog}}

A continuación se muestra el listado de historias de usuario (\textit{Product Backlog}) completo separadas por módulo, y para cada historia de usuario sus dependencias, estimación (en puntos de historia) y prioridad.

\begin{longtable} {r l c c c}
	\hline
	\#	&	\textbf{Descripción}					&	\textbf{Dep.}	&	\textbf{Est.}	&	\textbf{Prio.}	\\
	\hline \hline
	\endhead
	\multicolumn{5}{l}{\textbf{Auxiliar}} \\
	\hline 
	1.1.	&	Exportar volumen					&	-				&	12				&	1	\\
	\hline
	1.2.	&	Importar volumen					&	1.1.			&	8				&	5	\\
	\hline
	\multicolumn{5}{l}{\textbf{Pre-procesamiento}} \\
	\hline 
	2.1.	&	Filtro \textit{gaussiano}			&	-				&	24				&	1	\\
	\hline
	2.2.	&	Filtro media						&	-				&	10				&	3	\\
	\hline
	2.3.	&	Filtro mediana						&	-				&	8				&	3	\\
	\hline
	\multicolumn{5}{l}{\textbf{Segmentación}} \\
	\hline 
	3.1.	&	Segmentar pieza de madera			&	1.1.			&	64				&	1	\\
	\hline
	\multicolumn{5}{l}{\textbf{Documentación}} \\
	\hline 
	4.1.	&	Crear ROD							&	-				&	4				&	1	\\
	\hline
	4.2.	&	Eliminar ROD						&	4.1.			&	4				&	3	\\
	\hline
	4.3.	&	Exportar ROD						&	4.1.			&	2				&	2	\\
	\hline
	4.4.	&	Importar ROD						&	4.3.			&	3				&	2	\\
	\hline
	4.5.	&	Cambiar ROD							&	4.1.			&	3				&	1	\\
	\hline
	4.6.	&	Crear regla							&	-				&	3				&	2	\\
	\hline
	4.7.	&	Eliminar regla						&	4.6				&	3				&	4	\\
	\hline
	4.8.	&	Editar regla						&	4.6				&	1				&	5	\\
	\hline
	4.9.	&	Ocultar regla						&	4.6				&	2				&	6	\\
	\hline
	4.10.	&	Mostrar regla						&	4.9				&	1				&	6	\\
	\hline
	4.11.	&	Crear transportador de ángulos		&	-				&	3				&	3	\\
	\hline
	4.12.	&	Eliminar transportador de ángulos	&	4.11.			&	3				&	5	\\
	\hline
	4.13.	&	Editar transportador de ángulos		&	4.11.			&	1				&	6	\\
	\hline
	4.14.	&	Ocultar transportador de ángulos	&	4.11.			&	2				&	7	\\
	\hline
	4.15.	&	Mostrar transportador de ángulos	&	4.14.			&	1				&	7	\\
	\hline
	4.16.	&	Crear nota							&	-				&	4				&	2	\\
	\hline
	4.17.	&	Eliminar nota						&	4.16.			&	2				&	4	\\
	\hline
	4.18.	&	Editar nota							&	4.16.			&	3				&	5	\\
	\hline
	4.19.	&	Ocultar nota						&	4.16.			&	2				&	6	\\
	\hline
	4.20.	&	Mostrar nota						&	4.19.			&	1				&	6	\\
	\hline
	\multicolumn{5}{l}{\textbf{Mejoras en código}} \\
	\hline
	5.1.	&	Internacionalización				&	-				&	6				&	8	\\
	\hline
	\\
	\caption{Historias de usuario}
	\label{tab:analisis/hus}
\end{longtable}

Se han estimado un total de 180 PH (Puntos de historia) y se habían planificado unas 360 horas de trabajo. Por lo que cada PH corresponde aproximadamente a 2 horas.

Al haber registrado 312 horas finalmente, he determinado que mi velocidad ha sido aproximadamente 1,73 horas/PH, un poco más rápido que lo planificado.