\chapter{Especificación de requisitos}

Este capítulo es una Especificación de Requisitos Software para el software que se va a realizar siguiendo las directrices dadas por el estándar IEEE830 \cite{iee830}.

\section{Introducción}

\subsection{Propósito}

Este capítulo de especificación de requisitos tiene como objetivo definir las especificaciones funcionales y no funcionales para el desarrollo de un software que permitirá pre-procesar datos DICOM obtenidos al someter a una escultura de madera policromada a una TC para que posteriormente los usuarios puedan realizar labores de documentación al explorar la figura así como realizar una segmentación de los distintos trozos de madera que la componen. 

\subsection{Ámbito del sistema}

En la actualidad, los datos DICOM obtenidos tras una TC se utilizan, principalmente, en el campo donde surgieron, la medicina. Con este software llamado 3DCurator, se tratará de trasladar esta técnica al campo de la restauración de bienes culturales y poder pre-procesar, visualizar, interactuar y documentar los datos DICOM obtenidos con esculturas de madera policromada.